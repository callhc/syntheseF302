\subsection{Construction de formules}
Le vocabulaire du langage de la logique propositionnelle est composé de :
\begin{enumerate}
  \item de propositions $x$, $y$, $z$, ...; ou $X$, $Y$, $Z$, ...;
  \item de deux constantes vrai ($\top$ ou $1$) et faux ($\bot$ ou $0$);
  \item d'un ensemble de connecteurs logiques : $\neg$, $\wedge$, $\vee$, $\rightarrow$, $\leftrightarrow$.
  \item de paranthèses $(\ )$.
\end{enumerate}


\subsection{Sémantique}
\begin{definition}{Sémantique}{sémantique}
La sémantique d'une formule est la valeur de vérité de cette formule. La valeur de vérité d'une formule
$\Phi$ formée àpd propositions d'un ensemble $X$, évaluée avec la fonction d'interprétation $V$, est notée $\llbracket \Phi \rrbracket_V$.
La fonction $\llbracket \Phi \rrbracket_V$ est définie par induction sur la syntaxe de $\Phi$ de la façon suivante :
\begin{itemize}[label=$\bullet$]
  \item $\llbracket \top\rrbracket_V = 1$ ; $\llbracket \bot\rrbracket_V = 0$ ; $\llbracket x\rrbracket_V = V(x)$
  \item $\llbracket \neg \Phi\rrbracket_V = 1 - \llbracket \Phi\rrbracket_V$
  \item $\llbracket \Phi_1 \vee \Phi_2\rrbracket_V = \text{max}(\llbracket\Phi_1\rrbracket_V,\llbracket\Phi_2\rrbracket_V)$
  \item $\llbracket \Phi_1 \land \Phi_2\rrbracket_V = \text{min}(\llbracket\Phi_1\rrbracket_V,\llbracket\Phi_2\rrbracket_V)$
  \item $\llbracket \Phi_1 \rightarrow \Phi_2\rrbracket_V = \text{max}(1 - \llbracket\Phi_1\rrbracket_V,\llbracket\Phi_2\rrbracket_V)$
  \item $\llbracket \Phi_1 \leftrightarrow \Phi_2\rrbracket_V = \text{min}(\llbracket\Phi_1\rightarrow\Phi_2\rrbracket_V,\llbracket\Phi_2\rightarrow\Phi_1\rrbracket_V)$
\end{itemize}
Nous notons $V\vDash\Phi\Leftrightarrow\llbracket\Phi\rrbracket_V=1$ soit "$V$ satisfait $\Phi$."
\end{definition}

L'information contenue dans la définition est souvent représentée sous forme de table de verité : 
\begin{center}
  \begin{tabular}{|c|c|c|c|c|c|}
    \hline 
    $\Phi_1$ & $\Phi_2$ & $\Phi_1\vee\Phi_2$ & $\Phi_1\wedge\Phi_2$ & $\Phi_1\rightarrow\Phi_2$ & $\Phi_1\leftrightarrow\Phi_2$ \\ 
    \hline 
    0 & 0 & 0 & 0 & 1 & 1 \\ 
    \hline 
    0 & 1 & 1 & 0 & 1 & 0 \\ 
    \hline 
    1 & 0 & 1 & 0 & 0 & 0 \\ 
    \hline 
    1 & 1 & 1 & 1 & 1 & 1 \\ 
    \hline 
  \end{tabular}
\end{center}
\warningbox{Dans l'implication suivante : $\Phi_1\rightarrow\Phi_2$, la cas où $\Phi_1$ est faux ne nous intéresse pas. Dans ce cas, l'implication est toujours vraie.
  Une hypothèse vraie ne peut pas mener à une conclusion fausse.
}

\begin{example}\leavevmode
  Prenons $\Phi$ la formule suivante : $\Phi = (x\vee y) \wedge (\neg y \vee z)$
  Considérons alors la fonction d'interprétation $V_1$ telle que $V_1(x)=1$, $V_1(y)=0$ et $V_1(z)=1$.
  On a alors : 
  \begin{align*}
    \llbracket\Phi\rrbracket_{V_1} &= \llbracket(x\vee y)\wedge(\neg y\vee z)\rrbracket_{V_1} \\
                                   &= (x\vee y) \wedge (\neg y \vee z) \\ 
                                   &= (1\vee 0) \wedge (\neg 0 \vee 1) \\ 
                                   &= 1 \wedge 1 \\ 
                                   &= 1
  \end{align*}
  On a donc $V_1\vDash\Phi$. 

  Si on considère maintenant la fonction d'interprétation $V_2$ telle que $V_2(x)=0$, $V_2(y)=0$ et $V_2(z)=1$, on a :
  \begin{align*}
    \llbracket\Phi\rrbracket_{V_2} &= \llbracket(x\vee y)\wedge(\neg y\vee z)\rrbracket_{V_2} \\
                                   &= (x\vee y) \wedge (\neg y \vee z) \\ 
                                   &= (0\vee 0) \wedge (\neg 0 \vee 1) \\ 
                                   &= 0 \wedge 1 \\ 
                                   &= 0 
  \end{align*} 
  On a donc $V_2\nvDash\Phi$.
\end{example}

\subsection{Validité et Stabilité}
\subsubsection{Définitions}
\begin{definition}{Formule propositionnelle satisfaisable}{formule_propositionnelle_satisfaisable}
Une formule propositionnelle $\Phi$ est \textbf{satisfaisable} $\Leftrightarrow$ \textbf{il existe} une fonction d'interprétation $V$ pour les propositions de $\Phi$, telle que 
$V\vDash\Phi$.
\end{definition}
\begin{definition}{Formule propositionnelle valide}{formule_propositionnelle_valide}
Une formule propositionnelle $\Phi$ est \textbf{valide} $\Leftrightarrow$ \textbf{pour toute} fonction d'interprétation $V$ pour les propositions de $\Phi$, $V\vDash\Phi$.
\end{definition}

\begin{example}\leavevmode
  En reprenant l'exemple précédent, on a que $\Phi$ est satisfaisable car $V_1\vDash\Phi$. Cependant, $\Phi$ n'est pas valide car $V_2\nvDash\Phi$.

  Notons alors $\Phi_1$ la formule $\Phi_1 = \neg(x\vee y) \iff (\neg x \wedge \neg y)$.
  On a alors que $\Phi_1$ est valide car $\Phi_1$ est vraie pour toutes les fonctions d'interprétation. 
  En effet, c'est un loi de Morgan.
\end{example}

\subsubsection{Conséquence logique}
\begin{definition}{Conséquence Logique}{conséquence_logique}
Soit $\Phi_1,...,\Phi_n,\Phi$ des formules. On dira que $\Phi$ est une \textbf{conséquence logique} de $\Phi_1,...,\Phi_n$, noté $\Phi_1,...,\Phi_n\vDash\Phi$, si ($\Phi_1\land ...\land\Phi_n$)$\rightarrow\Phi$ est valide.
\end{definition}

\begin{example}\leavevmode
  Prenons $p, \neg p \vDash \bot$. En effet, on a que $p\land\neg p$ est toujours faux.

\end{example}

\subsubsection{Equivalence}

\begin{definition}{Formules équivalentes}{formules_équivalentes}
Deux formules, $\Phi$ et $\Psi$, sont \textbf{équivalentes} si la formule $\Phi\leftrightarrow\Psi$ est valide. On notera $\Phi\equiv\Psi$.
\end{definition}
Pour toutes formules $\Phi_1,\Phi_2,\Phi_3$ :
\begin{itemize}[label=$\bullet$]
  \item $\neg\neg\Phi_1\equiv\Phi_1$
  \item $\neg(\Phi_1\land\Phi_2)\equiv(\neg\Phi_1\lor\neg\Phi_2)$
  \item $\neg(\Phi_1\lor\Phi_2)\equiv(\neg\Phi_1\land\neg\Phi_2)$
  \item $\Phi_1\land(\Phi_2\lor\Phi_3)\equiv(\Phi_1\land\Phi_2)\lor(\Phi_1\land\Phi_3)$
  \item $\Phi_1\lor(\Phi_2\land\Phi_3)\equiv(\Phi_1\lor\Phi_2)\land(\Phi_1\lor\Phi_3)$
  \item $\Phi_1\rightarrow\Phi_2\equiv(\neg\Phi_1\lor\Phi_2)$
  \item $\Phi_1\leftrightarrow\Phi_2\equiv(\Phi_1\rightarrow\Phi_2)\land(\Phi_2\rightarrow\Phi_1)$
\end{itemize}

\newpage
Les relations d'équivalence nous permettent de susbstituer des formules par d'autres équivalentes dans une formule.
Cela nous permet de simplifier des formules.

Pour démontrer que deux formules sont équivalentes, on peut utiliser les tables de vérité. 
\begin{example}\leavevmode
  Prenons $\Phi_1 \vee (\Phi_2 \wedge \Phi_3) \equiv \Phi_1 \vee \Phi_2) \wedge (\Phi_1 \vee \Phi_3)$.
  Il nous suffit de vérifier que $\Phi_1 \vee (\Phi_2 \wedge \Phi_3) \iff \Phi_1 \vee \Phi_2) \wedge (\Phi_1 \vee \Phi_3)$ est valide 
  pour toutes les fonctions d'interprétation ($\Phi_1, \Phi_2, \Phi_3$). 
  Il faut donc vérifier que les 2 formules ont la même table de vérité (même valeur de vérité pour toutes les fonctions d'interprétation).
  Voici la table de vérité: 
  \begin{center}
    \begin{tabular}{|c|c|c|c|c|c|}
      \hline 
      $\Phi_1$ & $\Phi_2$ & $\Phi_3$ & $\Phi_1\vee(\Phi_2\wedge\Phi_3)$ & $(\Phi_1\vee\Phi_2)\wedge(\Phi_1\vee\Phi_3)$ \\ 
      \hline 
      0 & 0 & 0 & 0 & 0 \\ 
      \hline 
      0 & 0 & 1 & 0 & 0 \\ 
      \hline 
      0 & 1 & 0 & 0 & 0 \\ 
      \hline 
      0 & 1 & 1 & 1 & 1 \\ 
      \hline 
      1 & 0 & 0 & 1 & 1 \\ 
      \hline 
      1 & 0 & 1 & 1 & 1 \\ 
      \hline 
      1 & 1 & 0 & 1 & 1 \\ 
      \hline 
      1 & 1 & 1 & 1 & 1 \\ 
      \hline 
    \end{tabular} 
  \end{center} 
  On voit que les deux formules ont la même table de vérité. Elles sont donc équivalentes. 
\end{example}


\subsubsection{Lien entre satisfaisabilité et validité}
\begin{theorem}{Lien entre satisfaisabilité et validité}{lien_satisfaisabilité_validité}
  Une formule $\Phi$ est valide $\Leftrightarrow$ $\neg\Phi$ est insatisfaisable.
\end{theorem}
En effet, ça signifie que pour toute fonciton d'interprétation, il n'existe pas de fonction d'interprétation qui satisfait $\neg\Phi$.
S'il y en avait une, alors $\Phi$ ne serait pas valide car il y aurait une entrée pour laquelle $\Phi$ est fausse.
\begin{figure}[H]
  \centering
  \includegraphics[scale=0.3]{pictures/satisf:vali.png}
  \caption{Lien entre satisfaisabilité et validité}
\end{figure}


\subsubsection{Tableaux sémantiques}
\begin{definition}{Littéral}{littéral}
  Un littéral est une proposition $x$ ou la négation d'une proposition $\neg x$.
\end{definition}

\begin{theorem}{Satisfaisabilité des littéraux}{satisfaisabilité_litéraux}
  Un ensemble $S$ de littéraux est satisfaisable \textbf{ssi} il ne contient pas de littéraux et leur négation, (pair de littéraux \textbf{complémentaire}, $x$ et $\neg x$).
\end{theorem}
\begin{example}\leavevmode
  $\{x, \neg x\}$ est insatisfaisable. alors que $\{x, y, \neg z\}$ est satisfaisable.
\end{example}
La méthode des tableaux sémantiques est un algorithme pour tester la satisfaisabilité d'une formule. Elle consiste à construire un arbre dont les noeuds sont des formules et les feuilles sont des littéraux. On construit l'arbre de la façon suivante :
\begin{itemize}[label=$\bullet$]
  \item On place la formule à tester à la racine de l'arbre.
  \item On applique les règles suivantes jusqu'à ce que l'arbre soit complet :
  \begin{itemize}[label=$\circ$]
    \item Si la formule à tester est une constante, on arrête.
    \item Si la formule à tester est une conjonction, on ajoute les deux conjoncteurs comme fils de la formule à tester.
    \item Si la formule à tester est une disjonction, on ajoute un fils avec le premier disjoncteur et un autre fils avec le deuxième disjoncteur.
    \item Si la formule à tester est une implication, on ajoute un fils avec la négation de l'antécédent et un autre fils avec le conséquent.
    \item Si la formule à tester est une équivalence, on ajoute un fils avec la négation de la première formule et un autre fils avec la deuxième formule.
    \item Si la formule à tester est une négation, on ajoute un fils avec la négation de la formule à tester.
  \end{itemize}
\end{itemize}
\begin{remark}
  TODO --> Vérifier l'algorithme
\end{remark}
\begin{figure}[H]
  \begin{center}
    \includegraphics[width=0.5\textwidth]{../pictures/tableausem.png}
  \end{center}
  \caption{Exemple de l'arbre créé avec l'algorithme}\label{fig:tableausem}
\end{figure}

