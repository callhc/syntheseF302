\documentclass{rapportULB}

\usepackage{tcolorbox}
\tcbuselibrary{minted,breakable,xparse,skins}
\usepackage{url}
\usepackage{lipsum}
\usepackage{enumitem}
\usepackage{hyperref}
\usepackage{awesomebox}
\usepackage{color}
\usepackage{amsmath}
\usepackage{algorithm}
\usepackage{algpseudocode}
\usepackage[nottoc,numbib]{tocbibind}
\usepackage{tabularx}
\usepackage{svg}
\usepackage{rotating}
\usepackage{float}
\usepackage{ulem}
\usepackage{stmaryrd}
\usepackage{amssymb}
\usepackage{mathtools}    % 


\definecolor{bg}{gray}{0.95}
\DeclareTCBListing{mintedbox}{O{}m!O{}}{%
  breakable=true,
  listing engine=minted,
  listing only,
  minted language=#2,
  minted style=default,
  minted options={%
    linenos,
    gobble=0,
    breaklines=true,
    breakafter=,,
    fontsize=\small,
    numbersep=8pt,
    #1},
  boxsep=0pt,
  left skip=0pt,
  right skip=0pt,
  left=25pt,
  right=0pt,
  top=3pt,
  bottom=3pt,
  arc=5pt,
  leftrule=0pt,
  rightrule=0pt,
  bottomrule=2pt,
  toprule=2pt,
  colback=bg,
  colframe=orange!70,
  enhanced,
  overlay={%
    \begin{tcbclipinterior}
    \fill[orange!20!white] (frame.south west) rectangle ([xshift=20pt]frame.north west);
    \end{tcbclipinterior}},
  #3}

\newcommand\eindent{\endgroup}


\makeatletter
    \newcommand{\pushright}[1]{\ifmeasuring@#1\else\omit\hfill$\displaystyle#1$\fi\ignorespaces}
    \newcommand{\pushleft}[1]{\ifmeasuring@#1\else\omit$\displaystyle#1$\hfill\fi\ignorespaces}
\makeatother

\title{infoFond} %Titre du fichier

\begin{document}

%----------- Informations du rapport ---------

\titre{Synthèse} %Titre du fichier .pdf
\UE{INFO-F302} %Nom de la UE
\sujet{Informatique Fondamentale} %Nom du sujet

\enseignant{E. \textsc{Filiot}} %Nom des enseignants

\eleves{Hugo \textsc{Callens}} %Nom des élèves

%----------- Initialisation -------------------
        
\fairemarges %Afficher les marges
\fairepagedegarde %Créer la page de garde

%------------ Corps du rapport ----------------

%------------ Chapitre 1 ----------------------
\section{Logique propositionnelle}
\subsection{Construction de formules}
Le vocabulaire du langage de la logique propositionnelle est composé de :
\begin{enumerate}
  \item de propositions $x$, $y$, $z$, ...; ou $X$, $Y$, $Z$, ...;
  \item de deux constantes vrai ($\top$ ou $1$) et faux ($\bot$ ou $0$);
  \item d'un ensemble de connecteurs logiques : $\neg$, $\wedge$, $\vee$, $\rightarrow$, $\leftrightarrow$.
  \item de parantheses $(\ )$.
\end{enumerate}


\subsection{Sémantique}
\tipbox{La sémantique d'une formule est la valeur de vérité de cette formule. La valeur de vérité d'une formule
$\Phi$ formée àpd propositions d'un ensemble $X$, évaluée avec la fonction d'interprétation $V$, est notée $\llbracket \Phi \rrbracket_V$.
La fonction $\llbracket \Phi \rrbracket_V$ est définie par induction sur la syntaxe de $\Phi$ de la façon suivante :
\begin{itemize}[label=$\bullet$]
  \item $\llbracket \top\rrbracket_V = 1$ ; $\llbracket \bot\rrbracket_V = 0$ ; $\llbracket x\rrbracket_V = V(x)$
  \item $\llbracket \neg \Phi\rrbracket_V = 1 - \llbracket \Phi\rrbracket_V$
  \item $\llbracket \Phi_1 \vee \Phi_2\rrbracket_V = \text{max}(\llbracket\Phi_1\rrbracket_V,\llbracket\Phi_2\rrbracket_V)$
  \item $\llbracket \Phi_1 \land \Phi_2\rrbracket_V = \text{min}(\llbracket\Phi_1\rrbracket_V,\llbracket\Phi_2\rrbracket_V)$
  \item $\llbracket \Phi_1 \leftarrow \Phi_2\rrbracket_V = \text{max}(1 - \llbracket\Phi_1\rrbracket_V,\llbracket\Phi_2\rrbracket_V)$
  \item $\llbracket \Phi_1 \leftrightarrow \Phi_2\rrbracket_V = \text{min}(\llbracket\Phi_1\rightarrow\Phi_2\rrbracket_V,\llbracket\Phi_2\rightarrow\Phi_1\rrbracket_V)$
\end{itemize}
Nous notons $V\vDash\Phi\Leftrightarrow\llbracket\Phi\rrbracket_V=1$ soit "$V$ satisfait $\Phi$."}

L'information contenue dans la définition est souvent représentée sous forme de table de verité : 
\begin{center}
  \begin{tabular}{|c|c|c|c|c|c|}
    \hline 
    $\Phi_1$ & $\Phi_2$ & $\Phi_1\vee\Phi_2$ & $\Phi_1\wedge\Phi_2$ & $\Phi_1\rightarrow\Phi_2$ & $\Phi_1\leftrightarrow\Phi_2$ \\ 
    \hline 
    0 & 0 & 0 & 0 & 1 & 1 \\ 
    \hline 
    0 & 1 & 1 & 0 & 1 & 0 \\ 
    \hline 
    1 & 0 & 1 & 0 & 0 & 0 \\ 
    \hline 
    1 & 1 & 1 & 1 & 1 & 1 \\ 
    \hline 
  \end{tabular}
\end{center}
\warningbox{Dans l'implication suivante : $\Phi_1\rightarrow\Phi_2$, la cas où $\Phi_1$ est faux ne nous intéresse pas. Dans ce cas, l'implication est toujours vraie.}



\subsection{Validité et Stabilité}
\subsubsection{Définitions}
\tipbox{Une formule propositionnelle $\Phi$ est \textbf{satisfaisable} $\Leftrightarrow$ il existe une fonction d'interprétation $V$ pour les propositions de $\Phi$, telle que 
$V\vDash\Phi$.}
\tipbox{Une formule propositionnelle $\Phi$ est \textbf{valide} $\Leftrightarrow$ pour toute fonction d'interprétation $V$ pour les propositions de $\Phi$, $V\vDash\Phi$.}

\subsubsection{Conséquence logique}
\tipbox{Soit $\Phi_1,...,\Phi_n,\Phi$ des formules. On dira que $\Phi$ est une \textbf{conséquence logique} de $\Phi_1,...,\Phi_n$, noté $\Phi_1,...,\Phi_n\vDash\Phi$, si ($\Phi_1\land ...\land\Phi_n$)$\rightarrow\Phi$ est valide.}

\subsubsection{Equivalence}
\tipbox{Deux formules, $\Phi$ et $\Psi$, sont \textbf{équivalentes} si la formule $\Phi\leftrightarrow\Psi$ est valide. On notera $\Phi\equiv\Psi$.}
Pour toutes formules $\Phi_1,\Phi_2,\Phi_3$ :
\begin{itemize}[label=$\bullet$]
  \item $\neg\neg\Phi_1\equiv\Phi_1$
  \item $\neg(\Phi_1\land\Phi_2)\equiv(\neg\Phi_1\lor\neg\Phi_2)$
  \item $\neg(\Phi_1\lor\Phi_2)\equiv(\neg\Phi_1\land\neg\Phi_2)$
  \item $\Phi_1\land(\Phi_2\lor\Phi_3)\equiv(\Phi_1\land\Phi_2)\lor(\Phi_1\land\Phi_3)$
  \item $\Phi_1\lor(\Phi_2\land\Phi_3)\equiv(\Phi_1\lor\Phi_2)\land(\Phi_1\lor\Phi_3)$
  \item $\Phi_1\rightarrow\Phi_2\equiv(\neg\Phi_1\lor\Phi_2)$
\end{itemize}

\subsubsection{Lien entre satisfaisabilité et validité}
\importantbox{Une formule $\Phi$ est valide $\Leftrightarrow$ $\neg\Phi$ est insatisfaisable.}
\begin{figure}[H]
  \centering
  \includegraphics[scale=0.5]{pictures/satisf:vali.png}
  \caption{Lien entre satisfaisabilité et validité}
\end{figure}


\subsubsection{Tableaux sémantiques}
\tipbox{Un littéral est une proposition $x$ ou la négation d'une proposition $\neg x$.}
La méthode des tableaux sémantiques est un algorithme pour tester la satisfaisabilité d'une formule. Elle consiste à construire un arbre dont les noeuds sont des formules et les feuilles sont des littéraux. On construit l'arbre de la façon suivante :
\begin{itemize}[label=$\bullet$]
  \item On place la formule à tester à la racine de l'arbre.
  \item On applique les règles suivantes jusqu'à ce que l'arbre soit complet :
  \begin{itemize}[label=$\circ$]
    \item Si la formule à tester est une constante, on arrête.
    \item Si la formule à tester est une conjonction, on ajoute les deux conjoncteurs comme fils de la formule à tester.
    \item Si la formule à tester est une disjonction, on ajoute un fils avec le premier disjoncteur et un autre fils avec le deuxième disjoncteur.
    \item Si la formule à tester est une implication, on ajoute un fils avec la négation de l'antécédent et un autre fils avec le conséquent.
    \item Si la formule à tester est une équivalence, on ajoute un fils avec la négation de la première formule et un autre fils avec la deuxième formule.
    \item Si la formule à tester est une négation, on ajoute un fils avec la négation de la formule à tester.
  \end{itemize}
\end{itemize}

\end{document}