\documentclass[a4paper, 12pt]{extarticle}

\input{preamble.tex} 


% Based on 'Fun Template 1', available at https://www.overleaf.com/latex/templates/fun-template-1/drwvdzsrpgzz


\begin{document}

%================= Settings front page =================
\titre{Synthèse} %Titre du fichier .pdf
\UE{INFO-F302} %Nom de la UE
\sujet{Informatique Fondamentale} %Nom du sujet

\enseignant{E. \textsc{Filiot}} %Nom des enseignants

\eleves{Hugo \textsc{Callens}} %Nom des élèves
% \maketitle
\makemargins %Afficher les marges
\makefrontpage
\maketoc

%=======================================================

% this is the orginal latex code of the template
% \input{example}

%================= Content =============================


%========================= Logique propositionnelle =========================
\section{Logique propositionnelle}
\subsection{Construction de formules}
Le vocabulaire du langage de la logique propositionnelle est composé de :
\begin{enumerate}
  \item de propositions $x$, $y$, $z$, ...; ou $X$, $Y$, $Z$, ...;
  \item de deux constantes vrai ($\top$ ou $1$) et faux ($\bot$ ou $0$);
  \item d'un ensemble de connecteurs logiques : $\neg$, $\wedge$, $\vee$, $\rightarrow$, $\leftrightarrow$.
  \item de paranthèses $(\ )$.
\end{enumerate}


\subsection{Sémantique}
\begin{definition}{Sémantique}{sémantique}
La sémantique d'une formule est la valeur de vérité de cette formule. La valeur de vérité d'une formule
$\Phi$ formée àpd propositions d'un ensemble $X$, évaluée avec la fonction d'interprétation $V$, est notée $\llbracket \Phi \rrbracket_V$.
La fonction $\llbracket \Phi \rrbracket_V$ est définie par induction sur la syntaxe de $\Phi$ de la façon suivante :
\begin{itemize}[label=$\bullet$]
  \item $\llbracket \top\rrbracket_V = 1$ ; $\llbracket \bot\rrbracket_V = 0$ ; $\llbracket x\rrbracket_V = V(x)$
  \item $\llbracket \neg \Phi\rrbracket_V = 1 - \llbracket \Phi\rrbracket_V$
  \item $\llbracket \Phi_1 \vee \Phi_2\rrbracket_V = \text{max}(\llbracket\Phi_1\rrbracket_V,\llbracket\Phi_2\rrbracket_V)$
  \item $\llbracket \Phi_1 \land \Phi_2\rrbracket_V = \text{min}(\llbracket\Phi_1\rrbracket_V,\llbracket\Phi_2\rrbracket_V)$
  \item $\llbracket \Phi_1 \leftarrow \Phi_2\rrbracket_V = \text{max}(1 - \llbracket\Phi_1\rrbracket_V,\llbracket\Phi_2\rrbracket_V)$
  \item $\llbracket \Phi_1 \leftrightarrow \Phi_2\rrbracket_V = \text{min}(\llbracket\Phi_1\rightarrow\Phi_2\rrbracket_V,\llbracket\Phi_2\rightarrow\Phi_1\rrbracket_V)$
\end{itemize}
Nous notons $V\vDash\Phi\Leftrightarrow\llbracket\Phi\rrbracket_V=1$ soit "$V$ satisfait $\Phi$."
\end{definition}

L'information contenue dans la définition est souvent représentée sous forme de table de verité : 
\begin{center}
  \begin{tabular}{|c|c|c|c|c|c|}
    \hline 
    $\Phi_1$ & $\Phi_2$ & $\Phi_1\vee\Phi_2$ & $\Phi_1\wedge\Phi_2$ & $\Phi_1\rightarrow\Phi_2$ & $\Phi_1\leftrightarrow\Phi_2$ \\ 
    \hline 
    0 & 0 & 0 & 0 & 1 & 1 \\ 
    \hline 
    0 & 1 & 1 & 0 & 1 & 0 \\ 
    \hline 
    1 & 0 & 1 & 0 & 0 & 0 \\ 
    \hline 
    1 & 1 & 1 & 1 & 1 & 1 \\ 
    \hline 
  \end{tabular}
\end{center}
\warningbox{Dans l'implication suivante : $\Phi_1\rightarrow\Phi_2$, la cas où $\Phi_1$ est faux ne nous intéresse pas. Dans ce cas, l'implication est toujours vraie.}



\subsection{Validité et Stabilité}
\subsubsection{Définitions}
\begin{definition}{Formule propositionnelle satisfaisable}{formule_propositionnelle_satisfaisable}
Une formule propositionnelle $\Phi$ est \textbf{satisfaisable} $\Leftrightarrow$ il existe une fonction d'interprétation $V$ pour les propositions de $\Phi$, telle que 
$V\vDash\Phi$.
\end{definition}
\begin{definition}{Formule propositionnelle valide}{formule_propositionnelle_valide}
Une formule propositionnelle $\Phi$ est \textbf{valide} $\Leftrightarrow$ pour toute fonction d'interprétation $V$ pour les propositions de $\Phi$, $V\vDash\Phi$.
\end{definition}

\subsubsection{Conséquence logique}
\begin{definition}{Conséquence Logique}{conséquence_logique}
Soit $\Phi_1,...,\Phi_n,\Phi$ des formules. On dira que $\Phi$ est une \textbf{conséquence logique} de $\Phi_1,...,\Phi_n$, noté $\Phi_1,...,\Phi_n\vDash\Phi$, si ($\Phi_1\land ...\land\Phi_n$)$\rightarrow\Phi$ est valide.
\end{definition}

\subsubsection{Equivalence}

\begin{definition}{Formules équivalentes}{formules_équivalentes}
Deux formules, $\Phi$ et $\Psi$, sont \textbf{équivalentes} si la formule $\Phi\leftrightarrow\Psi$ est valide. On notera $\Phi\equiv\Psi$.
\end{definition}
Pour toutes formules $\Phi_1,\Phi_2,\Phi_3$ :
\begin{itemize}[label=$\bullet$]
  \item $\neg\neg\Phi_1\equiv\Phi_1$
  \item $\neg(\Phi_1\land\Phi_2)\equiv(\neg\Phi_1\lor\neg\Phi_2)$
  \item $\neg(\Phi_1\lor\Phi_2)\equiv(\neg\Phi_1\land\neg\Phi_2)$
  \item $\Phi_1\land(\Phi_2\lor\Phi_3)\equiv(\Phi_1\land\Phi_2)\lor(\Phi_1\land\Phi_3)$
  \item $\Phi_1\lor(\Phi_2\land\Phi_3)\equiv(\Phi_1\lor\Phi_2)\land(\Phi_1\lor\Phi_3)$
  \item $\Phi_1\rightarrow\Phi_2\equiv(\neg\Phi_1\lor\Phi_2)$
\end{itemize}

\subsubsection{Lien entre satisfaisabilité et validité}
\begin{theorem}{Lien entre satisfaisabilité et validité}{lien_satisfaisabilité_validité}
  Une formule $\Phi$ est valide $\Leftrightarrow$ $\neg\Phi$ est insatisfaisable.
\end{theorem}
\begin{figure}[H]
  \centering
  \includegraphics[scale=0.3]{pictures/satisf:vali.png}
  \caption{Lien entre satisfaisabilité et validité}
\end{figure}


\subsubsection{Tableaux sémantiques}
\begin{definition}{Littéral}{littéral}
  Un littéral est une proposition $x$ ou la négation d'une proposition $\neg x$.
\end{definition}
La méthode des tableaux sémantiques est un algorithme pour tester la satisfaisabilité d'une formule. Elle consiste à construire un arbre dont les noeuds sont des formules et les feuilles sont des littéraux. On construit l'arbre de la façon suivante :
\begin{itemize}[label=$\bullet$]
  \item On place la formule à tester à la racine de l'arbre.
  \item On applique les règles suivantes jusqu'à ce que l'arbre soit complet :
  \begin{itemize}[label=$\circ$]
    \item Si la formule à tester est une constante, on arrête.
    \item Si la formule à tester est une conjonction, on ajoute les deux conjoncteurs comme fils de la formule à tester.
    \item Si la formule à tester est une disjonction, on ajoute un fils avec le premier disjoncteur et un autre fils avec le deuxième disjoncteur.
    \item Si la formule à tester est une implication, on ajoute un fils avec la négation de l'antécédent et un autre fils avec le conséquent.
    \item Si la formule à tester est une équivalence, on ajoute un fils avec la négation de la première formule et un autre fils avec la deuxième formule.
    \item Si la formule à tester est une négation, on ajoute un fils avec la négation de la formule à tester.
  \end{itemize}
\end{itemize}
\begin{remark}
  algorithme généré par copilot.
\end{remark}






%========================= Déduction naturelle =========================
\section{Déduction naturelle}
\subsection{Règles}
\subsubsection{Règles pour la conjonction}
\begin{itemize}[label=$\bullet$]
  \item Règle d'introduction :
  \begin{equation*}
    \frac{\Phi\quad\Psi}{\Phi\land\Psi}\bigwedge i
  \end{equation*}
  \item Règle d'élimination :
  \begin{equation*}
    \frac{\Phi\land\Psi}{\Phi}\bigwedge e_1\quad\frac{\Phi\land\Psi}{\Psi}\bigwedge e_2
  \end{equation*}
\end{itemize}
\subsubsection{Règles pour la double négation}
\begin{itemize}[label=$\bullet$]
  \item Règle d'introduction :
  \begin{equation*}
    \frac{\Phi}{\neg\neg\Phi}\neg\neg i
  \end{equation*}
  \item Règle d'élimination :
  \begin{equation*}
    \frac{\neg\neg\Phi}{\Phi}\neg\neg e
  \end{equation*}
\end{itemize}
\subsubsection{Elimination de l'implication : Modus Ponens}
Règle d'élimination :
\begin{equation*}
  \frac{\Phi\quad\Phi\rightarrow\Psi}{\Psi}\rightarrow_\text{MP} (\text{ou} \rightarrow_e)
\end{equation*}
En contraposition, nous avons le Modus Tollens :
\begin{equation*}
  \frac{\Phi\rightarrow\Psi\quad\neg\Psi}{\neg\Phi}\rightarrow_\text{MT}
\end{equation*}
\subsubsection{Règle pour l'introduction de l'implication}
\begin{align*}
  &\Phi\text{hyp.} \\
  &...\\
  &\frac{\Psi\text{fin hyp.}}{\Phi\rightarrow\Psi}\rightarrow_i
\end{align*}

\subsubsection{Règle pour l'ouverture et la fermeture d'hypothèses}
\begin{itemize}[label=$\bullet$]
  \item toute hypothèse introduite doit être fermée.
  \item on ne peut jamais fermer deux hypothèses en même temps.
  \item Une fois une hypothèse fermée, on ne peut pas utilsier les formules déduites entre l'ouverture et la fermeture de cette hypothèse.
\end{itemize}


\subsubsection{Règle pour l'introduction de la disjonction}
\begin{equation*}
  \frac{\Phi}{\Phi\lor\Psi}\lor_{i_2}\quad\frac{\Psi}{\Phi\lor\Psi}\lor_{i_2}
\end{equation*}

\subsubsection{Elimination de la disjonction}
  \begin{align*}
    &\Psi_1 \text{hyp.} \quad \Psi_2 \text{hyp.} \\
    &...\\
    &\frac{\Psi_1\lor\Psi_2\quad\Phi\text{fin hyp.}\quad\Phi\text{fin hyp.}}{\Phi}\lor_e
  \end{align*}


\begin{remark}
  ... Il y a encore tout un tas de règles dérivées etc, mais je pense qu'elles sont principalement d'orde pratique. De même je pense que ce chapitre de cours demande plutôt de l'entrainement que de la théorie.
\end{remark}

\subsection{Théorèmes}
\begin{definition}{adéquation}{adéquation}
  Pour tout $\Psi_1,...,\Psi_n,\Psi$, si $\Psi_1,...,\Psi_n\vdash\Psi$ alors $\Psi_1,...,\Psi_n\vDash\Psi$.
\end{definition}
\begin{definition}{complétude}{complétude}
  Pour tout $\Psi_1,...,\Psi_n,\Psi$, si $\Psi_1,...,\Psi_n\vDash\Psi$ alors $\Psi_1,...,\Psi_n\vdash\Psi$.
\end{definition}





%================= Bibliography ========================
% \newpage
% \phantomsection % Required if hyperref is used
% \addcontentsline{toc}{section}{References} % Adding bibliography to table of contents
% \printbibliography % Print the bibliography

\end{document}
