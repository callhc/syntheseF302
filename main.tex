\documentclass[a4paper, 12pt]{extarticle}

\input{preamble.tex} 


% Based on 'Fun Template 1', available at https://www.overleaf.com/latex/templates/fun-template-1/drwvdzsrpgzz


\begin{document}

%================= Settings front page =================
\titre{Synthèse} %Titre du fichier .pdf
\UE{INFO-F302} %Nom de la UE
\sujet{Informatique Fondamentale} %Nom du sujet

\enseignant{E. \textsc{Filiot}} %Nom des enseignants

\eleves{Hugo \textsc{Callens}} %Nom des élèves
% \maketitle
\makemargins %Afficher les marges
\makefrontpage
\maketoc

%=======================================================

% this is the orginal latex code of the template
% \input{example}

%================= Content =============================


%========================= Logique propositionnelle =========================
\section{Logique propositionnelle}
\label{sec:logique_propositionnelle}
\subsection{Construction de formules}
Le vocabulaire du langage de la logique propositionnelle est composé de :
\begin{enumerate}
  \item de propositions $x$, $y$, $z$, ...; ou $X$, $Y$, $Z$, ...;
  \item de deux constantes vrai ($\top$ ou $1$) et faux ($\bot$ ou $0$);
  \item d'un ensemble de connecteurs logiques : $\neg$, $\wedge$, $\vee$, $\rightarrow$, $\leftrightarrow$.
  \item de paranthèses $(\ )$.
\end{enumerate}


\subsection{Sémantique}
\begin{definition}{Sémantique}{sémantique}
La sémantique d'une formule est la valeur de vérité de cette formule. La valeur de vérité d'une formule
$\Phi$ formée àpd propositions d'un ensemble $X$, évaluée avec la fonction d'interprétation $V$, est notée $\llbracket \Phi \rrbracket_V$.
La fonction $\llbracket \Phi \rrbracket_V$ est définie par induction sur la syntaxe de $\Phi$ de la façon suivante :
\begin{itemize}[label=$\bullet$]
  \item $\llbracket \top\rrbracket_V = 1$ ; $\llbracket \bot\rrbracket_V = 0$ ; $\llbracket x\rrbracket_V = V(x)$
  \item $\llbracket \neg \Phi\rrbracket_V = 1 - \llbracket \Phi\rrbracket_V$
  \item $\llbracket \Phi_1 \vee \Phi_2\rrbracket_V = \text{max}(\llbracket\Phi_1\rrbracket_V,\llbracket\Phi_2\rrbracket_V)$
  \item $\llbracket \Phi_1 \land \Phi_2\rrbracket_V = \text{min}(\llbracket\Phi_1\rrbracket_V,\llbracket\Phi_2\rrbracket_V)$
  \item $\llbracket \Phi_1 \rightarrow \Phi_2\rrbracket_V = \text{max}(1 - \llbracket\Phi_1\rrbracket_V,\llbracket\Phi_2\rrbracket_V)$
  \item $\llbracket \Phi_1 \leftrightarrow \Phi_2\rrbracket_V = \text{min}(\llbracket\Phi_1\rightarrow\Phi_2\rrbracket_V,\llbracket\Phi_2\rightarrow\Phi_1\rrbracket_V)$
\end{itemize}
Nous notons $V\vDash\Phi\Leftrightarrow\llbracket\Phi\rrbracket_V=1$ soit "$V$ satisfait $\Phi$."
\end{definition}

L'information contenue dans la définition est souvent représentée sous forme de table de verité : 
\begin{center}
  \begin{tabular}{|c|c|c|c|c|c|}
    \hline 
    $\Phi_1$ & $\Phi_2$ & $\Phi_1\vee\Phi_2$ & $\Phi_1\wedge\Phi_2$ & $\Phi_1\rightarrow\Phi_2$ & $\Phi_1\leftrightarrow\Phi_2$ \\ 
    \hline 
    0 & 0 & 0 & 0 & 1 & 1 \\ 
    \hline 
    0 & 1 & 1 & 0 & 1 & 0 \\ 
    \hline 
    1 & 0 & 1 & 0 & 0 & 0 \\ 
    \hline 
    1 & 1 & 1 & 1 & 1 & 1 \\ 
    \hline 
  \end{tabular}
\end{center}
\warningbox{Dans l'implication suivante : $\Phi_1\rightarrow\Phi_2$, la cas où $\Phi_1$ est faux ne nous intéresse pas. Dans ce cas, l'implication est toujours vraie.
  Une hypothèse vraie ne peut pas mener à une conclusion fausse.
}

\begin{example}\leavevmode
  Prenons $\Phi$ la formule suivante : $\Phi = (x\vee y) \wedge (\neg y \vee z)$
  Considérons alors la fonction d'interprétation $V_1$ telle que $V_1(x)=1$, $V_1(y)=0$ et $V_1(z)=1$.
  On a alors : 
  \begin{align*}
    \llbracket\Phi\rrbracket_{V_1} &= \llbracket(x\vee y)\wedge(\neg y\vee z)\rrbracket_{V_1} \\
                                   &= (x\vee y) \wedge (\neg y \vee z) \\ 
                                   &= (1\vee 0) \wedge (\neg 0 \vee 1) \\ 
                                   &= 1 \wedge 1 \\ 
                                   &= 1
  \end{align*}
  On a donc $V_1\vDash\Phi$. 

  Si on considère maintenant la fonction d'interprétation $V_2$ telle que $V_2(x)=0$, $V_2(y)=0$ et $V_2(z)=1$, on a :
  \begin{align*}
    \llbracket\Phi\rrbracket_{V_2} &= \llbracket(x\vee y)\wedge(\neg y\vee z)\rrbracket_{V_2} \\
                                   &= (x\vee y) \wedge (\neg y \vee z) \\ 
                                   &= (0\vee 0) \wedge (\neg 0 \vee 1) \\ 
                                   &= 0 \wedge 1 \\ 
                                   &= 0 
  \end{align*} 
  On a donc $V_2\nvDash\Phi$.
\end{example}

\subsection{Validité et Stabilité}
\subsubsection{Définitions}
\begin{definition}{Formule propositionnelle satisfaisable}{formule_propositionnelle_satisfaisable}
Une formule propositionnelle $\Phi$ est \textbf{satisfaisable} $\Leftrightarrow$ \textbf{il existe} une fonction d'interprétation $V$ pour les propositions de $\Phi$, telle que 
$V\vDash\Phi$.
\end{definition}
\begin{definition}{Formule propositionnelle valide}{formule_propositionnelle_valide}
Une formule propositionnelle $\Phi$ est \textbf{valide} $\Leftrightarrow$ \textbf{pour toute} fonction d'interprétation $V$ pour les propositions de $\Phi$, $V\vDash\Phi$.
\end{definition}

\begin{example}\leavevmode
  En reprenant l'exemple précédent, on a que $\Phi$ est satisfaisable car $V_1\vDash\Phi$. Cependant, $\Phi$ n'est pas valide car $V_2\nvDash\Phi$.

  Notons alors $\Phi_1$ la formule $\Phi_1 = \neg(x\vee y) \iff (\neg x \wedge \neg y)$.
  On a alors que $\Phi_1$ est valide car $\Phi_1$ est vraie pour toutes les fonctions d'interprétation. 
  En effet, c'est un loi de Morgan.
\end{example}

\subsubsection{Conséquence logique}
\begin{definition}{Conséquence Logique}{conséquence_logique}
Soit $\Phi_1,...,\Phi_n,\Phi$ des formules. On dira que $\Phi$ est une \textbf{conséquence logique} de $\Phi_1,...,\Phi_n$, noté $\Phi_1,...,\Phi_n\vDash\Phi$, si ($\Phi_1\land ...\land\Phi_n$)$\rightarrow\Phi$ est valide.
\end{definition}

\begin{example}\leavevmode
  Prenons $p, \neg p \vDash \bot$. En effet, on a que $p\land\neg p$ est toujours faux.

\end{example}

\subsubsection{Equivalence}

\begin{definition}{Formules équivalentes}{formules_équivalentes}
Deux formules, $\Phi$ et $\Psi$, sont \textbf{équivalentes} si la formule $\Phi\leftrightarrow\Psi$ est valide. On notera $\Phi\equiv\Psi$.
\end{definition}
Pour toutes formules $\Phi_1,\Phi_2,\Phi_3$ :
\begin{itemize}[label=$\bullet$]
  \item $\neg\neg\Phi_1\equiv\Phi_1$
  \item $\neg(\Phi_1\land\Phi_2)\equiv(\neg\Phi_1\lor\neg\Phi_2)$
  \item $\neg(\Phi_1\lor\Phi_2)\equiv(\neg\Phi_1\land\neg\Phi_2)$
  \item $\Phi_1\land(\Phi_2\lor\Phi_3)\equiv(\Phi_1\land\Phi_2)\lor(\Phi_1\land\Phi_3)$
  \item $\Phi_1\lor(\Phi_2\land\Phi_3)\equiv(\Phi_1\lor\Phi_2)\land(\Phi_1\lor\Phi_3)$
  \item $\Phi_1\rightarrow\Phi_2\equiv(\neg\Phi_1\lor\Phi_2)$
  \item $\Phi_1\leftrightarrow\Phi_2\equiv(\Phi_1\rightarrow\Phi_2)\land(\Phi_2\rightarrow\Phi_1)$
\end{itemize}

\newpage
Les relations d'équivalence nous permettent de susbstituer des formules par d'autres équivalentes dans une formule.
Cela nous permet de simplifier des formules.

Pour démontrer que deux formules sont équivalentes, on peut utiliser les tables de vérité. 
\begin{example}\leavevmode
  Prenons $\Phi_1 \vee (\Phi_2 \wedge \Phi_3) \equiv \Phi_1 \vee \Phi_2) \wedge (\Phi_1 \vee \Phi_3)$.
  Il nous suffit de vérifier que $\Phi_1 \vee (\Phi_2 \wedge \Phi_3) \iff \Phi_1 \vee \Phi_2) \wedge (\Phi_1 \vee \Phi_3)$ est valide 
  pour toutes les fonctions d'interprétation ($\Phi_1, \Phi_2, \Phi_3$). 
  Il faut donc vérifier que les 2 formules ont la même table de vérité (même valeur de vérité pour toutes les fonctions d'interprétation).
  Voici la table de vérité: 
  \begin{center}
    \begin{tabular}{|c|c|c|c|c|c|}
      \hline 
      $\Phi_1$ & $\Phi_2$ & $\Phi_3$ & $\Phi_1\vee(\Phi_2\wedge\Phi_3)$ & $(\Phi_1\vee\Phi_2)\wedge(\Phi_1\vee\Phi_3)$ \\ 
      \hline 
      0 & 0 & 0 & 0 & 0 \\ 
      \hline 
      0 & 0 & 1 & 0 & 0 \\ 
      \hline 
      0 & 1 & 0 & 0 & 0 \\ 
      \hline 
      0 & 1 & 1 & 1 & 1 \\ 
      \hline 
      1 & 0 & 0 & 1 & 1 \\ 
      \hline 
      1 & 0 & 1 & 1 & 1 \\ 
      \hline 
      1 & 1 & 0 & 1 & 1 \\ 
      \hline 
      1 & 1 & 1 & 1 & 1 \\ 
      \hline 
    \end{tabular} 
  \end{center} 
  On voit que les deux formules ont la même table de vérité. Elles sont donc équivalentes. 
\end{example}


\subsubsection{Lien entre satisfaisabilité et validité}
\begin{theorem}{Lien entre satisfaisabilité et validité}{lien_satisfaisabilité_validité}
  Une formule $\Phi$ est valide $\Leftrightarrow$ $\neg\Phi$ est insatisfaisable.
\end{theorem}
En effet, ça signifie que pour toute fonciton d'interprétation, il n'existe pas de fonction d'interprétation qui satisfait $\neg\Phi$.
S'il y en avait une, alors $\Phi$ ne serait pas valide car il y aurait une entrée pour laquelle $\Phi$ est fausse.
\begin{figure}[H]
  \centering
  \includegraphics[scale=0.3]{pictures/satisf:vali.png}
  \caption{Lien entre satisfaisabilité et validité}
\end{figure}


\subsubsection{Tableaux sémantiques}
\begin{definition}{Littéral}{littéral}
  Un littéral est une proposition $x$ ou la négation d'une proposition $\neg x$.
\end{definition}

\begin{theorem}{Satisfaisabilité des littéraux}{satisfaisabilité_litéraux}
  Un ensemble $S$ de littéraux est satisfaisable \textbf{ssi} il ne contient pas de littéraux et leur négation, (pair de littéraux \textbf{complémentaire}, $x$ et $\neg x$).
\end{theorem}
\begin{example}\leavevmode
  $\{x, \neg x\}$ est insatisfaisable. alors que $\{x, y, \neg z\}$ est satisfaisable.
\end{example}
La méthode des tableaux sémantiques est un algorithme pour tester la satisfaisabilité d'une formule. Elle consiste à construire un arbre dont les noeuds sont des formules et les feuilles sont des littéraux. On construit l'arbre de la façon suivante :
\begin{itemize}[label=$\bullet$]
  \item On place la formule à tester à la racine de l'arbre.
  \item On applique les règles suivantes jusqu'à ce que l'arbre soit complet :
  \begin{itemize}[label=$\circ$]
    \item Si la formule à tester est une constante, on arrête.
    \item Si la formule à tester est une conjonction, on ajoute les deux conjoncteurs comme fils de la formule à tester.
    \item Si la formule à tester est une disjonction, on ajoute un fils avec le premier disjoncteur et un autre fils avec le deuxième disjoncteur.
    \item Si la formule à tester est une implication, on ajoute un fils avec la négation de l'antécédent et un autre fils avec le conséquent.
    \item Si la formule à tester est une équivalence, on ajoute un fils avec la négation de la première formule et un autre fils avec la deuxième formule.
    \item Si la formule à tester est une négation, on ajoute un fils avec la négation de la formule à tester.
  \end{itemize}
\end{itemize}
\begin{remark}
  TODO --> Vérifier l'algorithme
\end{remark}
\begin{figure}[H]
  \begin{center}
    \includegraphics[width=0.5\textwidth]{../pictures/tableausem.png}
  \end{center}
  \caption{Exemple de l'arbre créé avec l'algorithme}\label{fig:tableausem}
\end{figure}


\newpage


%========================= Déduction naturelle =========================
\section{Déduction naturelle}
\label{sec:déduction_naturelle}
\begin{definition}{Déduction naturelle}{déduction_naturelle}
    La déduction naturelle est un système de preuve pour la logique propositionnelle. Il est composé de règles d'inférence qui permettent de déduire de nouvelles formules à partir de formules existantes. Une preuve est un arbre dont les noeuds sont des formules et les feuilles sont des axiomes. Une preuve est correcte si elle respecte les règles d'inférence. Une preuve est complète si elle contient toutes les formules qui sont des conséquences logiques des axiomes.
\end{definition}
\subsection{Règles pour la conjonction}
\begin{itemize}[label=$\bullet$]
\item Règle d'introduction :
\begin{equation*}
    \frac{\Phi\quad\Psi}{\Phi\land\Psi}\bigwedge i
\end{equation*}
\item Règle d'élimination :
\begin{equation*}
    \frac{\Phi\land\Psi}{\Phi}\bigwedge e_1\quad\frac{\Phi\land\Psi}{\Psi}\bigwedge e_2
\end{equation*}
\end{itemize}
\begin{example}
La règle d'introduction se lit : si j'ai une preuve de $\Phi$ et une preuve de $\Psi$, alors j'ai une preuve de $\Phi\land\Psi$.
\end{example}

\subsection{Règles pour la double négation}
\begin{itemize}[label=$\bullet$]
\item Règle d'introduction :
\begin{equation*}
    \frac{\Phi}{\neg\neg\Phi}\neg\neg i
\end{equation*}
\item Règle d'élimination :
\begin{equation*}
    \frac{\neg\neg\Phi}{\Phi}\neg\neg e
\end{equation*}
\end{itemize}


\subsection{Elimination de l'implication : Modus Ponens}
Règle d'élimination :
\begin{equation*}
\frac{\Phi\quad\Phi\rightarrow\Psi}{\Psi}\rightarrow_\text{MP} (\text{ou} \rightarrow_e)
\end{equation*}
\begin{example} 
$\;$
\begin{enumerate}
    \item $\Phi$ : "Il pleut"
    \item $\Psi$ : "S'il pleut, "je prends mon parapluie"
    \item Alors on en déduit que "je prends mon parapluie"
\end{enumerate}
\end{example}

En contraposition, nous avons le Modus Tollens :
\begin{equation*}
\frac{\Phi\rightarrow\Psi\quad\neg\Psi}{\neg\Phi}\rightarrow_\text{MT}
\end{equation*}
\begin{example}
$\;$
\begin{enumerate}
    \item $\Phi$ : "s'il pleut, alors la route est mouillée"
    \item $\Psi$ : "la route n'est pas mouillée"
    \item Alors on en déduit que "il ne pleut pas"
\end{enumerate}
\end{example}

\subsection{Règle pour l'introduction de l'implication}
\begin{align*}
&\Phi\text{hyp.} \\
&...\\
&\frac{\Psi\text{fin hyp.}}{\Phi\rightarrow\Psi}\rightarrow_i
\end{align*}
\warningbox{Lorsqu'on posera une hypoèse, on indentera l'hypothèse et toutes les lignes de la sous-preuve, jusqu'à la fermeture d'hypothèse.}
\begin{example}
    Ici, on va voir un exemple de ce qu'on a vu jusque maintenant :\newline
    On veut démontrer : $t \vdash (t\rightarrow p)\rightarrow(q\rightarrow(s\rightarrow p))$ \\
    Ici le prémisse est "t est vrai", le prémisse sera toujours ce qui se trouve à gauche de la déduction.\\
    \begin{align*}
        &1. \quad t \quad\quad\quad\quad\quad\quad\quad\quad \text{prémisse}.\\
        &2. \quad\quad t\rightarrow p \quad\quad\quad\quad\quad\quad\quad\quad \text{hyp}_1.\\
        &3. \quad\quad\quad q \quad\quad\quad\quad\quad\quad\quad\quad\quad\quad\quad \text{hyp}_2.\\
        &4. \quad\quad\quad\quad s\rightarrow p \quad\quad\quad\quad\quad\quad\quad\quad\quad \text{hyp}_3.\\
        &5. \quad\quad\quad\quad\quad p \quad\quad\quad\quad\quad\quad\quad\quad\quad\quad\quad \text{MP(1,2),fin hyp}_3.\\
        &6. \quad\quad\quad\quad s\rightarrow p \quad\quad\quad\quad\quad\quad\quad\quad\quad \rightarrow_i\text{(4,5),fin hyp}_2.\\
        &7. \quad\quad\quad q\rightarrow(s\rightarrow p) \quad\quad\quad\quad\quad\quad \rightarrow_i\text{(3,6),fin hyp}_1.\\
        &8. \quad (t\rightarrow p)\rightarrow(q\rightarrow(s\rightarrow p)) \quad\quad \rightarrow_i\text{(2,7)}.
    \end{align*}

    Nous pouvons également prouver des formules sans prémisse comme suit :
    \begin{align*}
        &1. \quad\quad p \quad\quad\quad\quad\quad\quad\quad\quad\quad \text{hyp}\\
        &2. \quad\quad\quad\quad \neg\neg p \quad\quad\quad\quad\quad\quad \neg\neg_i(1),\text{fin hyp.}\\
        &3. \quad\quad p\rightarrow\neg\neg p \quad\quad\quad\quad\quad \rightarrow_i(1,2).
    \end{align*}
    On a établit que $\vdash p \rightarrow\neg\neg p$.
    \begin{remark}
        Les formules $\Phi$ telles que : $\vdash\Phi$ sont appelées des théorèmes.
    \end{remark}
\end{example}



\subsection{Règle pour l'ouverture et la fermeture d'hypothèses}
\begin{itemize}[label=$\bullet$]
\item toute hypothèse introduite doit être fermée.
\item on ne peut jamais fermer deux hypothèses en même temps.
\item Une fois une hypothèse fermée, on ne peut pas utilsier les formules déduites entre l'ouverture et la fermeture de cette hypothèse.
\end{itemize}


\subsection{Règle pour l'introduction de la disjonction}
\begin{equation*}
\frac{\Phi}{\Phi\lor\Psi}\lor_{i_2}\quad\frac{\Psi}{\Phi\lor\Psi}\lor_{i_2}
\end{equation*}

\subsection{Elimination de la disjonction}
\begin{align*}
    &\Psi_1 \text{hyp.} \quad \Psi_2 \text{hyp.} \\
    &...\quad\quad\quad ...\\
    &\frac{\Psi_1\lor\Psi_2\quad\Phi\text{fin hyp.}\quad\Phi\text{fin hyp.}}{\Phi}\lor_e
\end{align*}
\begin{example}
    Supposons que les faits suivants soient vrais : 
    \begin{enumerate}
        \item si ma note d'examen est excellente, j'irai boire un verre.
        \item si ma note d'examen est bonne, j'irai boire un verre.
        \item ma note sera excellente ou bonne.
    \end{enumerate}
    Alors je peux en déduire que j'irai boire un verre.
\end{example}
\warningbox{Ici aussi, on ne peut pas utiliser l'hypothèse temporaire faite pour l'autre cas. (sauf si elle a été établie avant)}

\subsection{Règle de copie}
\begin{equation*}
\frac{\Phi}{\Phi}\text{copie}
\end{equation*}

\subsection{Règle pour la négation}
Les contradictions sont des formules de la forme : 
\begin{align*}
    \neg\Phi\land\Phi\quad \text{ou}\quad \neg\Phi\land\neg\Phi
\end{align*}
Toutes les contradictions sont logiquement équivalentes à la formule $\bot$. (rappel: $\bot$ est une formule qui est toujours fausse) \newline
Le fait que l'on peut tout déduire à partir d'une contradiction est formalisé par la règle suivante :
\begin{equation*}
\frac{\bot}{\Phi}\bot_e
\end{equation*}
Le fait que $\bot$ représente une contradiction est formalisé par la règle suivante :
\begin{equation*}
\frac{\Phi\quad\neg\Phi}{\bot}\neg_e
\end{equation*}
Afin d'introduire une négation, supposons que l'on fasse une hypothèse et que l'on arrive à déduire une contradiction, dans ce cas, l'hypothèse est fausse. Ceci est formaisé par la règle suivante :
\begin{align*}
    &\Phi \quad \text{hyp.}\\
    &...\\
    &\frac{\bot\text{fin hyp.}}{\neg\Phi}\neg_i
\end{align*}


\subsection{Règles pour l'équivalence}
\begin{equation*}
    \frac{\Phi_1\rightarrow\Phi_2\quad\Phi_2\rightarrow\Phi_1}{\Phi_1\leftrightarrow\Phi_2}\leftrightarrow_i
\end{equation*}
\begin{equation*}
    \frac{\Phi_1\leftrightarrow\Phi_2}{\Phi_1\rightarrow\Phi_2}\leftrightarrow_{e_1}\quad\frac{\Phi_1\leftrightarrow\Phi_2}{\Phi_2\rightarrow\Phi_1}\leftrightarrow_{e_2}
\end{equation*}

\subsection{Règles dérivées}
Il existe de nombreuses formules dérivées qui peuvent s'obtenir à partir des autres règles vues plus haut. (à voir si beaucoup utilisée au TP, si oui, ajouter : MT,RAA,LEM)


\subsection{Théorèmes}
\begin{definition}{adéquation}{adéquation}
Pour tout $\Psi_1,...,\Psi_n,\Psi$, si $\Psi_1,...,\Psi_n\vdash\Psi$ alors $\Psi_1,...,\Psi_n\vDash\Psi$.
\end{definition}
\begin{definition}{complétude}{complétude}
Pour tout $\Psi_1,...,\Psi_n,\Psi$, si $\Psi_1,...,\Psi_n\vDash\Psi$ alors $\Psi_1,...,\Psi_n\vdash\Psi$.
\end{definition}

\subsection{Démontrer une implication}
Il existe deux méthodes pour démontrer une implication ($A\rightarrow B$):
\begin{enumerate}
    \item On suppose A et on en déduit B.
    \item On suppose non B et on en déduit non A.
\end{enumerate}

\subsection{Démontrer une équivalence}
Il existe deux méthodes pour démontrer une équivalence ($A\leftrightarrow B$):
\begin{enumerate}
    \item On suppose A et on en déduit B et réciproquement on suppose B et on en déduit A.
    \item On prouve une chaîne d'équivalences.
\end{enumerate}

\subsection{Preuve par cas}
Ce type de preuve repose sur une généralisation de la règle $\lor_e$ : si on sait qu'on est soit dans le cas $A_1$ soit dans le cas $A_n$, et que pout tout $i\in\{1,...,n\}$, on peut démontrer une propriété $P$, alors c'est que $P$ est vraie.

\subsection{Preuve par contradiction}
On veut démontrer une propriété $P$. On suppose son contraire $\neg P$ et on en déduit une contradiction. On en déduit que $P$ est vraie.

\begin{remark}
Cette partie du cours nécessite de prendre le temps de bien comprendre les exemples donnés dans le cours ainsi que les exercices vus au TP.
\end{remark}
\newpage


%========================= SAT =========================
\section{Le Problème SAT}
\label{sec:sat}
\begin{remark}
    Soit $P$ un ensemble de propositions. Une formule $\Phi$ de la logique propositionnelle sur $P$ est satisfaisable s'il exite une interprétation
    \begin{equation*}
        V : X \rightarrow \{1, 0\}, \text{telle que } V \models \Phi
    \end{equation*}
\end{remark}
\subsection{Littéraux et Clauses}
\begin{definition}{Littéral}{litteral}
    Un \textbf{littéral} est une variable $x$ ou sa négation $\neg x$.
\end{definition}
\begin{definition}{Clause}{clause}
    Une \textbf{clause} est une disjonction de littéraux $l_1 \vee l_2 \vee \dots \vee l_n$. Elle est satisfaite par une valuation $V$ s'il existe $i$ tel que $V(l_i) = 1$. Par extension, on appelle .
\end{definition}
\begin{example}
    $p$ et $\neg p$ sont des littéraux. $x \vee\neg y \vee r$ est une clause mais pas $x \wedge y$.
\end{example}
\begin{theorem}{Satisfaction d'ensemble de clauses}{satensembleclauses}
    Un ensemble de clauses $A = \{C_1, \dots, C_n\}$ est satisfait par une valuation $V$, notée $V \models A$, si pour tout $i$, $V \models C_i$. En particulier, tout valuation satisfait l'ensemble vide $A = \emptyset$.
\end{theorem}
\subsubsection{Lien avec les formes normales}
\begin{itemize}[label=\textbullet]
    \item Une formule est en forme normale conjonctive (\textbf{FNC}) si et seulement si c'est une conjonction de disjonctions de littéraux, de la forme :
    \begin{equation*}
        \bigwedge_i(\bigvee_j(\neg)x_{i,j})
    \end{equation*}
    \item Une formule est en forme normale disjonctive (\textbf{FND}) si et seulement si c'est une disjonction de conjonctions de littéraux, de la forme :
    \begin{equation*}
        \bigvee_i(\bigwedge_j(\neg)x_{i,j})
    \end{equation*}
\end{itemize}
\begin{lemma}{Equivalence formule FNC}{equiFNC}
    Tout ensemble non-vide de clauses $A = \{C_1,\dots,C_n\}$ est équivalent à la formule en FNC $\Phi_A = \bigwedge_{i=1}^{n} C_i$, au sens où pour toute valuation, $V \models A \Leftrightarrow V \models \Phi_A$.
\end{lemma}
\begin{remark}
    Pour mettre une formule sous forme de clauses, il suffit de la mettre en FNC.
\end{remark}
\subsubsection{Mise sous FNC et FND}\label{sec:mettreFNCFND}
Afin de parvenir à une FNC ou une FND, on utilise les transformations successives suivantes pour obtenir les formes normales :
\begin{enumerate}
    \item élimination des connecteurs $\rightarrow$ et $\leftrightarrow$ grâce aux équivalences suivantes :
    \begin{equation*}
        (\Phi\rightarrow\Psi)\equiv(\neg\Phi\vee\Psi)
    \end{equation*}
    \begin{equation*}
        (\Phi\leftrightarrow\Psi)\equiv(\neg\Psi\vee\Psi)\wedge(\Phi\vee\neg\Psi)
    \end{equation*}
    \item entrer les négations le plus à l'intérieur possible :
    \begin{equation*}
        \neg(\Phi\wedge\Psi)\equiv(\neg\Phi\vee\neg\Psi) \qquad \neg\neg\Phi \equiv \Psi
    \end{equation*}
    \begin{equation*}
        \neg(\Psi\vee\Psi)\equiv(\neg\Phi\wedge\Psi)
    \end{equation*}
    \item utilisation des distributivité de $\vee$ et $\wedge$ :
    \begin{equation*}
        (\Phi\vee(\Psi\wedge\chi )) \equiv (\Phi\vee\Psi)\wedge(\Phi\vee\chi)\text{(mise sous FNC)}
    \end{equation*}
    \begin{equation*}
        (\Phi\wedge(\Psi\vee\chi)) \equiv (\Phi\wedge\Psi)\vee(\Phi\wedge\chi)\text{(mise sous FND)}
    \end{equation*}
\end{enumerate}
\begin{example}
    Illustrons cela à travers un exemple, mettons la formule $\neg(x\leftrightarrow(y\rightarrow r))$
    \begin{enumerate}
        \item on retire les équivalences et implications :
        \begin{equation*}
            \neg((\neg x\vee\neg y \vee r)\wedge(\neg(\neg y\vee r)\vee x))
        \end{equation*}
        \item on pousse les négations à l'intérieur :
        \begin{equation*}
            (x\wedge y\wedge\neg r) \vee ((\neg y\vee r)\wedge\neg x)
        \end{equation*}
        \item on distribue : 
        \begin{equation*}
            (x\wedge y\wedge\neg r)\vee((\neg y\vee r)\wedge\neg x)
        \end{equation*}
        \item et encore : 
        \begin{equation*}
            (x\vee\neg y\vee r)\wedge(y\vee\neg y\vee r)\wedge(\neg r\vee\neg y\vee r)\wedge(x\vee\neg x)\wedge(y\vee\neg x)\wedge(\neg r\vee\neg x)
        \end{equation*}
        \item on retire les formules équivalentes à $\top$ :
        \begin{equation*}
            (x\vee\neg y\vee r)\wedge(y\vee\neg x)\wedge(\neg r\vee\neg x)
        \end{equation*}
    \end{enumerate}
    Invitation à prendre le temps de faire cette démonstrations sur une feuille étape par étape pour voir si tout a bine été compris.
\end{example}

\subsection{Problème SAT}
\begin{definition}{Problème SAT}{problemeSAT}
    Une \textbf{entrée} est un ensemble de clauses $S$. \\
    Une \textbf{sortie} est de la forme : Est-ce que $S$ est \textbf{sat}isfaisable?
\end{definition}
Si $S$ est satisfaisable, on aimerait que l'algorithme nous retourne une valuation $V$ qui satisfait $S$.
\begin{remark}
    Actuellement, on ne sait toujours pas s'il existe un algorithme pour ce problème dont la complexité en temps est polynomiale. L'intuition veut que ce ne soit pas le cas mais aucune preuve n'a pu être fournie jusqu'à présent. Si on parvenait à le prouver, on pourrait gagner un prix d'un million de dollars ;). C'est un problème qui appartient à la classe NP. (cf. chapitre sur la complexité) 
\end{remark}

\subsection{Introduction aux solveurs SAT}
\begin{definition}{Solveur SAT}{solveurSAT}
    Un solveur est un programme qui décidé le problème SAT. Si la formule est satisfaisable, une interprétation qui la satifsait est retournée. Ils ont une complexité au pire cas exponentielle. 
\end{definition}

\subsubsection{Notations pour les grandes disjonctions et conjonctions}
\begin{definition}{disjonction et conjonction}{disjonctionconjonction}
    \begin{itemize}[label=\textbullet]
        \item On appelera \textit{disjonction} une formule de la forme $\Phi_1\vee\Phi_2\vee\dots\vee\Phi_n$ où $n \geq 1$. Attention, lorsque $n=1$, la disjonction est réduite à la seule forme $\Phi_1$. On utilisera l'abbréviation suivante :
        \begin{equation*}
            \bigvee_{i=1}^{n}\Phi_i
        \end{equation*}
        \item On appelera \textit{conjonction} une formule de la forme $\Phi_1\wedge\Phi_2\wedge\dots\wedge\Phi_n$ où $n \geq 1$. Attention, lorsque $n=1$, la conjonction est réduite à la seule forme $\Phi_1$. On utilisera l'abbréviation suivante :
        \begin{equation*}
            \bigwedge_{i=1}^{n}\Phi_i
        \end{equation*}
    \end{itemize}
    
\end{definition}

\subsection{Modélisation}
Dans cette section, nous feront la modélisation du problème des 8 reines, d'autres exemples sont présents dans le cours mais le raisonnement demeure identique.
\begin{figure}[H]
    \centering
    \includegraphics[scale=0.3]{pictures/prob8reines.png}
    \caption{Problème des 8 reines}
    \label{fig:8reines}
\end{figure}
\subsubsection{Choix des variables}
\begin{equation*}
    X = \{x_{i,j} | i,j \in \{1,\dots,8\}\}
\end{equation*}
La sémantique est la suivante : "$x_{i,j}$ est vraie si et seulement s'il y a une reine en case $(i,j)$".

\subsubsection{Expression des contraintes}
\begin{enumerate}
    \item Pas d'attaque horizontale
    \item Pas d'attaque verticale
    \item Pas d'attaque en diagonale
    \item Au moins une reine par ligne
\end{enumerate}

\subsubsection*{Pas d'attaque horizontale}
"Pour toute ligne, il n'existe pas deux reines sur cette ligne."
\begin{equation*}
    \bigwedge_{i \in \{1,\dots,8\}} \neg (\bigvee_{j,j' \in \{1,\dots,8\} | j \neq j'} x_{i,j} \wedge x_{i,k})
\end{equation*}
\begin{equation*}
    \equiv \bigwedge_{i,j,j' \in \{1,\dots,8\} | j \neq j'} (\neg x_{i,j} \vee \neg x_{i,j'})
\end{equation*}
\subsubsection*{Pas d'attaque verticale}
"Pour toute colonne, il n'existe pas deux reines sur cette colonne."
\begin{equation*}
    \bigwedge_{j \in \{1,\dots,8\}} \neg (\bigvee_{i,i' \in \{1,\dots,8\} | i \neq i'} x_{i,j} \wedge x_{k,j})
\end{equation*}
\begin{equation*}
    \equiv \bigwedge_{i,i',j \in \{1,\dots,8\} | i \neq i'} (\neg x_{i,j} \vee \neg x_{i',j})
\end{equation*}
\subsubsection*{Pas d'attaque en diagonale}
"Pour toute diagonale, il n'existe pas deux reines sur cette diagonale."
\begin{equation*}
    \bigwedge_{i,i',j,j' \in \{1,\dots,8\} | i \neq i' \wedge j \neq j'} \neg (x_{i,j} \wedge x_{i',j'} \wedge |i-i'| = |j-j'|)
\end{equation*}
\begin{equation*}
    \equiv \bigwedge_{i,i',j,j' \in \{1,\dots,8\} | i \neq i' \wedge j \neq j'} (\neg x_{i,j} \vee \neg x_{i',j'} \vee |i-i'| \neq |j-j'|)
\end{equation*}
\subsubsection*{Au moins une reine par ligne}
"Pour toute ligne, il existe au moins une reine sur cette ligne."
\begin{equation*}
    \bigwedge_{i \in \{1,\dots,8\}} (\bigvee_{j \in \{1,\dots,8\}} x_{i,j})
\end{equation*}

\subsection{Algorithme DPLL}
Ne sera pas à l'examen.

\subsection{Transformation de Tseitin}
Il se peut que le problème ne s'exprime pas facilement par une formule en FNC. Le but de la transformation de Tseitin est d'ajouter de nouvelles variables et des équivalences.
\begin{example}
    Considérons $\Phi = (x \wedge q) \vee \neg(y \vee r)$, Dans la transformation de Tseitin, on va remplacer $x \wedge q$ par une nouvelle variable $x_1$ et $\neg(y \vee r)$ par une nouvelle variable $x_2$. Voici la formule considérée :
    \begin{equation*}
        (x_1\vee x_2) \wedge (x_1 \leftrightarrow x\wedge q) \wedge (x_2 \leftrightarrow \neg(y\vee r))
    \end{equation*}
    Il reste encore à mattre les deux formules en FNC : (cf. \ref{sec:mettreFNCFND})
    \begin{itemize}[label=\textbullet]
        \item $x_1 \leftrightarrow x \wedge q$
        \begin{equation*}
            \begin{aligned}
                &\equiv (x_1 \rightarrow (x \wedge q)) \wedge ((x \wedge q) \rightarrow x_1) \\
                &\equiv (\neg x_1 \vee (x \wedge q)) \wedge (\neg(x \wedge q) \vee x_1) \\
                &\equiv (\neg x_1 \vee x) \wedge (\neg x_1 \vee q) \wedge (\neg x \vee \neg q \vee x_1)
            \end{aligned}
        \end{equation*}
        \item $x_2 \leftrightarrow \neg(y\vee r)$
        \begin{equation*}
            \begin{aligned}
                &\equiv (x_2 \rightarrow \neg(y\vee r)) \wedge (\neg(y\vee r) \rightarrow x_2) \\
                &\equiv (\neg x_2 \vee \neg y \wedge \neg r)\wedge (y \vee r \vee x_2) \\
                &\equiv (\neg x_2 \vee \neg y) \wedge (\neg x_2 \vee \neg r) \wedge (y \vee r \vee x_2)
            \end{aligned}
        \end{equation*}
    \end{itemize}
    Dès lors, nous pouvons écrire la formule sous FNC suivante :
    \begin{equation*}
        \Psi = (x_1\vee x_2)\wedge(\neg x_1 \vee p)\wedge(\neg x_1\vee q)\wedge(\neg x \vee \neg y \vee x_1)\wedge(\neg x_2 \vee \neg q) \wedge (\neg x_2 \vee \neg r) \wedge (y \vee r \vee x_2)
    \end{equation*}
\end{example}
La transformation de Tseitin est intéressante lorsqu'on devra mettre sous FNC des formules qui sont sous forme normale disjonctive $C_1\vee C_2\vee\dots\vee C_n$. Car il suffit d'introduire une variable $x_i$ pour chaque $C_i$ et on obtient la formule :
\begin{equation*}
    (x_1\vee x_2\vee\dots\vee x_n)\wedge(x_1\leftrightarrow C_1)\wedge\dots\wedge(x_n \leftrightarrow C_n)
\end{equation*} 
Si on suppose que $C_i = l_1\wedge\dots l_k$ où $l_i$ sont des littéraux, alors mettre $x_i\leftrightarrow C_i$ sous FNC est assez simple :
\begin{equation*}
    x_i \leftrightarrow C_i \equiv (\bigwedge_{j=1}^{k}(\neg x_i \vee l_j)) \wedge (x_i \vee \bigvee_{j=1}^{k}(\neg l_j))
\end{equation*}

\warningbox{Il ne faut surtout pas hésiter à ajouter des variables pour minimiser le nombre de clauses.}

\newpage

\section{Automates}
\label{sec:automates}
\subsection{Introduction}
\label{sub:introduction}
\begin{example}
    Un automate fini :
    \begin{itemize}[label=\textbullet]
        \item lit la séquence des lettres de gauche à droite.
        \item possède un nombre fini d'états.
        \item en fonction de la lettre courante et de la lettre lue, se déplace vers un autre état.
        \item possède un unique état initial ainsi que des états finaux.
        \item accepte un mot si et seulement s'il se termine sur un état final.
    \end{itemize}
    \begin{figure}[H]
        \centering
        \begin{tikzpicture} [node distance = 3cm, 
            on grid, 
            auto,
            every loop/.style={stealth-}]
        
        % State q0 
        \node (q0) [state, 
            initial, 
            accepting, 
            initial text = {}] {$q_0$};
        
        % State q1    
        \node (q1) [state,
            right = of q0] {$q_1$};
        
        % Arrows
        \path [-stealth, thick]
            (q0) edge[bend left] node {$a$}   (q1)
            (q1) edge[bend left] node {$a$}   (q0)
            (q0) edge [loop above]  node {b}()
            (q1) edge [loop above]  node {b}();
        \end{tikzpicture}
        \caption{Exemple d'automate fini}
    \end{figure}
\end{example}

\subsection{Définitions et exemples}
\label{sub:definitions_et_exemples}

\begin{definition}{Langage}{langage}
    \begin{itemize}[label=\textbullet]
        \item un alphabet est un ensemble fini, que l'on note $\Sigma$.
        \item ses éléments sont appelés lettres ou symboles.
        \item un mot est une suite de symboles, $\epsilon$ est le mot vide.
        \item l'ensemble des mots sur $\Sigma$ est noté $\Sigma^*$.
        \item un langage est un sous-ensemble de $\Sigma^*$. ($L \subseteq \Sigma^*$)
    \end{itemize}
\end{definition}
\begin{definition}{Automate fini}{automate_fini}
    Un automate fini $A$ sur un alphabet $\Sigma$ est un 4-uplet $(Q, q_0, F, \delta)$ où :
    \begin{itemize}[label=\textbullet]
        \item $Q$ est un ensemble fini d'états.
        \item $q_0 \in Q$ est l'état initial.
        \item $F \subseteq Q$ est l'ensemble des états finaux.
        \item $\delta : Q \times \Sigma \rightarrow Q$ est la \textbf{fonction} de transition.
    \end{itemize}
\end{definition}
\begin{definition}{Exécution}{exécution}
    Une exécution d'un automate $A$ est une suite finie $e = q_0\sigma_1 q_1\sigma_2 ... \sigma_n q_n(n\geq 0)$ telle que :
    \begin{itemize}[label=\textbullet]
        \item $\forall i \in \{0, ..., n\}, q_i \in Q$.
        \item $\forall i \in \{1, ..., n\}, \sigma_i \in \Sigma$.
        \item $\forall i \in \{0, ..., n-1\}, \delta(q_i, \sigma_{i+1}) = q_{i+1}$.
    \end{itemize}
    Une exéution $e$ est dite \textbf{acceptante} si $q_n \in F$.
\end{definition}
\begin{definition}{Langage accepté}{langage_accepté}
    Le langage accepté par un automate $A$ (noté $L(A)$) est l'ensemble des mots pour lesquels il existe une exécution
    acceptante de $A$.
    \begin{equation*}
        L(A) = \{w \in \Sigma^* | \exists e \text{ exécution acceptante de } A \text{ sur } w\}
    \end{equation*}
\end{definition}
\begin{example}
    Pour $\Sigma = \{0,1\} :$
    \begin{figure}[H]
        \centering
        \begin{tikzpicture} [node distance = 3cm, 
            on grid, 
            auto,
            every loop/.style={stealth-}]
        
        % State q0 
        \node (q0) [state, 
            initial,  
            initial text = {}] {$q_0$};
        
        % State q1    
        \node (q1) [state,
            right = of q0] {$q_1$};

        % State q2
        \node (q2) [state,
            right = of q1] {$q_2$};

        % State q3
        \node (q3) [state,
            accepting,
            right = of q2] {$q_3$};

        
        % Arrows
        \path [-stealth, thick]
            (q0) edge node {1}   (q1)
            (q1) edge node {0}   (q2)
            (q2) edge node {1}   (q3)
            (q2) edge [bend left] node {0} (q0)
            (q0) edge [loop above]  node {0}()
            (q1) edge [loop above]  node {1}()
            (q3) edge [loop above]  node {0,1}();
        \end{tikzpicture}
    \end{figure}
    Le langage accepté par cet automate est :
    \begin{equation*}
        L(A) = \{w\in \{0,1\}^* | w \text{ contient le facteur } 101\}
    \end{equation*}
\end{example}
\begin{definition}{Automate Complet}{automate_complet}
    Un automate $A$ est dit \textit{complet} si sa fonction de transition est totale.
\end{definition}
\begin{lemma}{Transformation d'un automate en un automate complet}{transformation_automate_complet}
    On peut toujours transformer un automate $A$ en un automate $B$ complet qui accepte le même langage, tel que $L(A) = L(B)$.  
\end{lemma}
\begin{remark}
    Effectivement, il \textit{suffit} de rajouter un état supplémentaire (état puit) non final et ajouter les transitions 
    manquantes vers cet état.
\end{remark}



\subsubsection{Test du vide}
Le problème du vide est le suivant :
\begin{itemize}
    \item Entrée $\Rightarrow$ Étant donné un automate $A$ sur un alphabet $\Sigma$.
    \item Sortie $\Rightarrow$ est-ce que $L(A) = \emptyset$ ?
\end{itemize}
\begin{definition}{Etats atteignables}{états_atteignables}
    Soit $A=(Q, q_0, F, \delta)$ un automate sur un alphabet $\Sigma$. Un état $q\in Q$ est dit \textit{atteignable} si 
    il existe un mot $w\in \Sigma^*$ et une exécution de $A$ sur $w$ qui termine en $q$.
\end{definition}
\begin{remark}
    L'ensemble des états atteignables d'un automate peut être calculé en temps $O(n+m)$, où $n$ est le nombre d'états et
    $m$ le nombre de transitions.
\end{remark}
\begin{theorem}{Test du vide}{test_vide}
    Étant donné un automate $A$ avec $n$ états et $m$ transitions, on peut tester en temps $O(n+m)$ si $L(A) = \emptyset$.
\end{theorem}
L'algorithme serait le suivant :
\begin{enumerate}
    \item Calculer l'ensemble des états atteignables $R$ de $A$.
    \item tester si $R \cap F = \emptyset$.
\end{enumerate}



\subsubsection{Opérations Booléennes}
\begin{definition}{Complément}{complément}
    Le complément d'un langage $L\subseteq \Sigma^*$ est le langage, noté $\overline{L}$, défini par :
    \begin{equation*}
        \overline{L} = \{w\in \Sigma^*|w\notin L\} = \Sigma^* \setminus L
    \end{equation*}
\end{definition}
\begin{example}
    Si $L$ est l'ensemble des mots sur $\{a,b\}$ qui contiennent au moins un $a$, alors $\overline{L}$ est l'ensemble des
    mots qui contiennet au moins deux $a$, ou pas de $a$.
\end{example}
\begin{definition}{Union et intersection}{union_intersection}
    Soient $L_1, L_2 \subseteq \Sigma^*$ deux langages. L'union et l'intersection de $L_1$ et $L_2$ sont définies par :
    \begin{equation*}
        L_1 \cup L_2 = \{w\in \Sigma^*|w\in L_1 \textbf{ ou } w\in L_2\}
    \end{equation*}
    \begin{equation*}
        L_1 \cap L_2 = \{w\in \Sigma^*|w\in L_1 \textbf{ et } w\in L_2\}
    \end{equation*}
\end{definition}
\begin{theorem}{Clôtures des automates par opératiions Booléennes}{clôture}
    Soient $A,A_1,A_2$ des automates finis sur un alphabet $\Sigma$. Il existe des automates $A_c, U, I$ tels que :
    \begin{equation*}
        L(A_c) = \overline{L(A)}
    \end{equation*}
    \begin{equation*}
        L(U) = L(A_1) \cup L(A_2)
    \end{equation*}
    \begin{equation*}
        L(I) = L(A_1) \cap L(A_2)
    \end{equation*}
\end{theorem}

\begin{example}
    Si $A=(Q, q_0, F, \delta)$ est complet (s'il n'est pas complet, il faut le compléter avant), il suffit de prendre
    $A_c = (Q, q_0, Q\setminus F, \delta)$.
    \begin{itemize}[label=\textbullet]
        \item Si $A=$
        \begin{figure}[H]
            \centering
            \begin{tikzpicture} [node distance = 3cm, 
                on grid, 
                auto,
                every loop/.style={stealth-}]
            
            % State q0 
            \node (q0) [state, 
                initial, 
                accepting, 
                initial text = {}] {$q_0$};
            
            % State q1    
            \node (q1) [state,
                right = of q0] {$q_1$};
            
            % Arrows
            \path [-stealth, thick]
                (q0) edge[bend left] node {$a$}   (q1)
                (q1) edge[bend left] node {$a$}   (q0)
                (q0) edge [loop above]  node {b}()
                (q1) edge [loop above]  node {b}();
            \end{tikzpicture}
        \end{figure}
        \item Alors, $A_c=$
        \begin{figure}[H]
            \centering
            \begin{tikzpicture} [node distance = 3cm, 
                on grid, 
                auto,
                every loop/.style={stealth-}]
            
            % State q0 
            \node (q0) [state, 
                initial, 
                initial text = {}] {$q_0$};
            
            % State q1    
            \node (q1) [state,
                accepting,
                right = of q0] {$q_1$};
            
            % Arrows
            \path [-stealth, thick]
                (q0) edge[bend left] node {$a$}   (q1)
                (q1) edge[bend left] node {$a$}   (q0)
                (q0) edge [loop above]  node {b}()
                (q1) edge [loop above]  node {b}();
            \end{tikzpicture}
        \end{figure}
    \end{itemize}
\end{example}
\begin{definition}{Produit d'automates}{produit_automates}
    Soient $A_1 = (Q_1, q_{01}, F_1, \delta_1)$ et $A_2 = (Q_2, q_{02}, F_2, \delta_2)$ deux automates sur un alphabet $\Sigma$.
    Le produit de $A_1$ et $A_2$ (noté $A_1\otimes A_2$) est le pré-automate défini par :
    \begin{equation*}
        A_1\otimes A_2 = (Q_1\times Q_2, (q_0^1, q_0^2), \delta_{12})
    \end{equation*}
    où, $\forall$ $(q_1, q_2) \in Q_1\times Q_2$ et $\forall$ $\sigma \in \Sigma$ :
    $$ \delta_{12}((q_1,q_2), \sigma) =
    \begin{cases}
        \text{indéfini} & \text{si} \delta_1(q_1,\sigma) \text{ est indéfinie} \\
        \text{indéfini} & \text{si} \delta_2(q_2,\sigma) \text {est indéfinie} \\
        (\delta_1(q_1,\sigma), \delta_2(q_2,\sigma)) & \text{sinon.}
    \end{cases} $$
\end{definition}
\begin{example}
    Voici un produit d'automates :
    \begin{figure}[H]
        \centering
        \begin{tikzpicture} [node distance = 3cm, 
            on grid, 
            auto,
            every loop/.style={stealth-}]
        
        % State q0 
        \node (q0) [state, 
            initial, 
            accepting,
            initial text = {$A_2$ : }] {$q_0$};
        
        % State q1    
        \node (q1) [state,
            right = of q0] {$q_1$};

        %state p0q0
        \node (p0q0) [state,
            initial,
            below = of q0,
            initial text = {}] {$p_0,q_0$};
        
        %state p0q1
        \node (p0q1) [state,
            accepting,
            right = of p0q0] {$p_0,q_1$};
        
        %state p1q0
        \node (p1q0) [state,
            below = of p0q0] {$p_1,q_0$};
        
        %state p1q1
        \node (p1q1) [state,
            right = of p1q0] {$p_1,q_1$};
        
        %state p0
        \node (p0) [state,
            initial,
            accepting,
            left = of p0q0,
            initial text = {$A_1$ : }] {$p_0$};
        
        %state p1
        \node (p1) [state,
            left = of p1q0] {$p_1$};
        
        % Arrows
        \path [-stealth, thick]
            (q0) edge[bend left] node {$a$}   (q1)
            (q1) edge[bend left] node {$a$}   (q0)
            (q0) edge [loop above]  node {b}()
            (q1) edge [loop above]  node {b}()
            (p0) edge[bend left] node {$a$}   (p1)
            (p1) edge[bend left] node {$a$}   (p0)
            (p0) edge [loop above]  node {b}()
            (p0q0) edge[bend left] node {a} (p1q1)
            (p1q1) edge[bend left] node {a} (p0q0)
            (p0q0) edge [loop above]  node {b}();
        \end{tikzpicture}
    \end{figure}
    \begin{itemize}[label=\textbullet]
        \item $L(A_1)$ est l'ensemble des mots qui contiennent un nombre pair de $b$.
        \item $L(A_2)$ est l'ensemble des mots qui contiennent un nombre pair de $a$.
        \item Toute exécution de $A_1\otimes A_2$ sur un mot $w$ simule ene parallèle l'exécution de $A_1$ sur $w$, ainsi
        que celles de $A_2$ sur $w$.
            \begin{itemize}
                \item pour $w=abba$
                \item sur $A_1$, on a $e_1 = p_0\; a\; p_0\; b\; p_1\; b\; p_0\; a\; p_0$.
                \item sur $A_2$, on a $e_2 = q_0\; a\; q_1\; b\; q_1\; b\; q_1\; a\; q_0$.
                \item sur $A_1\otimes A_2$, on a $e_{12} = (p_0, q_0)\; a\; (p_0, q_1)\; b\; (p_1, q_1)\; b\; (p_0, q_1)\; 
                a\; (p_0, q_0)$.
            \end{itemize}
        \item Pour avoir $L(A_1) \cap L(A_2)$, il suffit de prendre $F_{\cap} = F_1 \times F_2$ pour les états finaux de 
        $A_1\otimes A_2$.
        \item Pour avoir $L(A_1) \cup L(A_2)$, il suffit de prendre $F_{\cup} = (F_1 \times Q_2) \cup (Q_1 \times F_2)$ pour
        les états finaux de $A_1\otimes A_2$.
    \end{itemize}
\end{example}
\begin{theorem}{Clôture par union et intersection}{clôture_union_intersection}
    Soient $A_1 = (Q_1, q_{01}, F_1, \delta_1)$ et $A_2 = (Q_2, q_{02}, F_2, \delta_2)$ deux automates. Soit $A_1\otimes A_2 =
    (Q_1\times Q_2, (q_0^1,q_0^2),\delta_{12})$ le pré-automate produit.
    \begin{itemize}[label=\textbullet]
        \item Si $A_1$ et $A_2$ sont \textbf{complets} et \\
            $U = (Q_1\times Q_2, (q_0^1,q_0^2),(F_1\times Q_2)\cup(Q_1\times F_2),\delta_{12})$, alors
        \begin{equation*}
            L(U) = L(A_1) \cup L(A_2)
        \end{equation*}
        \item Si $I = (Q_1\times Q_2, (q_0^1,q_0^2),(F_1\times F_2),\delta_{12})$, alors
        \begin{equation*}
            L(I) = L(A_1) \cap L(A_2)
        \end{equation*}
    \end{itemize}
\end{theorem}
\begin{definition}{Automates équivalents}{automates_équivalents}
    Deux automates $A_1$ et $A_2$ sont \textbf{équivalents} si $L(A_1) = L(A_2)$.
\end{definition}
\begin{theorem}{Automates équivalents}{automates_équivalents_th}
    Étant donnés deux automates $A_1$ et $A_2$, il est décidable en temps polynomial si $L(A_1) \subseteq L(A_2)$, et si 
    $L(A_1) = L(A_2)$.
\end{theorem}
\begin{example}
    Soient $A_1$ et $A_2$ deux automates :
    \begin{figure}[H]
        \centering
        \begin{tikzpicture} [node distance = 3cm, 
            on grid, 
            auto,
            every loop/.style={stealth-}]
        
        % State q0 
        \node (q0) [state, 
            initial, 
            accepting, 
            initial text = {$A_1$ : }] {$p_0$};
        
        % State q1    
        \node (q1) [state,
            right = of q0] {$p_1$};
        
        % Arrows
        \path [-stealth, thick]
            (q0) edge[bend left] node {$a$}   (q1)
            (q1) edge[bend left] node {$a$}   (q0)
            (q0) edge [loop above]  node {b}();
        \end{tikzpicture}
        \hspace{2cm}
        \begin{tikzpicture} [node distance = 3cm, 
            on grid, 
            auto,
            every loop/.style={stealth-}]
        
        % State q0 
        \node (q0) [state, 
            initial, 
            accepting, 
            initial text = {$A_2$ : }] {$q_0$};
        
        % State q1    
        \node (q1) [state,
            right = of q0] {$q_1$};
        
        % Arrows
        \path [-stealth, thick]
            (q0) edge[bend left] node {$a$}   (q1)
            (q1) edge[bend left] node {$a$}   (q0)
            (q0) edge [loop above]  node {b}()
            (q1) edge [loop above]  node {b}();
        \end{tikzpicture}
    \end{figure}
    Est-ce que $L(A_1) \subseteq L(A_2)$ ? \\
    \begin{figure}[H]
        \centering
        \begin{tikzpicture} [node distance = 3cm, 
            on grid, 
            auto,
            every loop/.style={stealth-}]
        
        % State q0 
        \node (q0) [state, 
            initial, 
            initial text = {$\overline{A_2}$ : }] {$q_0$};
        
        % State q1    
        \node (q1) [state,
            accepting,
            right = of q0] {$q_1$};

        %state p0q0
        \node (p0q0) [state,
            initial,
            below = of q0,
            initial text = {}] {$p_0,q_0$};
        
        %state p0q1
        \node (p0q1) [state,
            accepting,
            right = of p0q0] {$p_0,q_1$};
        
        %state p1q0
        \node (p1q0) [state,
            below = of p0q0] {$p_1,q_0$};
        
        %state p1q1
        \node (p1q1) [state,
            right = of p1q0] {$p_1,q_1$};
        
        %state p0
        \node (p0) [state,
            initial,
            accepting,
            left = of p0q0,
            initial text = {$A_1$ : }] {$p_0$};
        
        %state p1
        \node (p1) [state,
            left = of p1q0] {$p_1$};
        
        % Arrows
        \path [-stealth, thick]
            (q0) edge[bend left] node {$a$}   (q1)
            (q1) edge[bend left] node {$a$}   (q0)
            (q0) edge [loop above]  node {b}()
            (q1) edge [loop above]  node {b}()
            (p0) edge[bend left] node {$a$}   (p1)
            (p1) edge[bend left] node {$a$}   (p0)
            (p0) edge [loop above]  node {b}()
            (p0q0) edge[bend left] node {a} (p1q1)
            (p1q1) edge[bend left] node {a} (p0q0)
            (p0q0) edge [loop above]  node {b}();
        \end{tikzpicture}
    \end{figure}
    Nous pouvons dès lors vérifier que $L(A_1) \cap \overline{L(A_2)} = \emptyset$ et que donc on a bien $L(A_1) \subseteq L(A_2)$. \\
    Les transitions sortantes de ($p_1,q_0$) et ($p_0,q_1$) ne sont pas utiles car ces états ne sont pas atteignables.
\end{example}




\subsection{Automates non-déterministes}
\label{sub:automates_non_déterministes}

\begin{example}
    Prenons l'alphabet $\Sigma = \{a,b\}$. Donner un automate qui accepte :
    \begin{equation*}
        L_3 = \{u \in \Sigma^* |\; |u|\geq 3 \text{ et } u[|u|-3] = a \}
    \end{equation*}
    Un automate déterministe répondant à cette question est le suivant :
    \begin{figure}[H]
        \centering
        \begin{tikzpicture} [node distance = 3cm, 
            on grid, 
            auto,
            every loop/.style={stealth-}]
        
        % State q0 
        \node (q0) [state, 
            initial, 
            initial text = {}] {$q_0$};
        
        % State q1    
        \node (q1) [state,
            accepting,
            right = of q0] {$q_1$};

        % State q2
        \node (q2) [state,
            above right = of q1] {$q_2$};
        
        % State q3
        \node (q3) [state,
            below right = of q1] {$q_3$};
            
        % State q5
        \node (q5) [state,
            accepting,
            right = of q2] {$q_5$};
            
        % State q4
        \node (q4) [state,
            accepting,
            above = of q5] {$q_4$};
            
        % State q6
        \node (q6) [state,
            accepting,
            right = of q3] {$q_6$};
        
        % State q7
        \node (q7) [state,
            accepting,
            below = of q6] {$q_7$};

        % Arrows
        \path [-stealth, thick]
            (q0) edge node {$a$}   (q1)
            (q0) edge [loop above]  node {$b$}()
            (q1) edge node {$a$}  (q2)
            (q1) edge node {$b$}  (q3)
            (q2) edge node {$a$} (q4)
            (q2) edge node {$b$} (q5)
            (q3) edge node {$a$} (q6)
            (q3) edge node {$b$} (q7)
            (q4) edge [loop above] node {$a$}()
            (q4) edge node {$b$} (q5)
            (q5) edge node {$a$} (q6)
            (q5) edge [bend left] node {$b$} (q7)
            (q6) edge node {$a$} (q2)
            (q6) edge [bend left] node {$b$} (q3)
            (q7) edge [bend left] node {$b$} (q0)
            (q7) edge [bend left] node {$a$} (q1);
        \end{tikzpicture}
    \end{figure}
    Nous pouvons le réduire en un automate non-déterministe :
    \begin{figure}[H]
        \centering
        \begin{tikzpicture} [node distance = 3cm, 
            on grid, 
            auto,
            every loop/.style={stealth-}]
        
        % State q0 
        \node (q0) [state, 
            initial, 
            initial text = {}] {$q_0$};
        
        % State q1    
        \node (q1) [state,
            right = of q0] {$q_1$};
        
        % State q2
        \node (q2) [state,
            right = of q1] {$q_2$};
        
        % State q3
        \node (q3) [state,
            accepting,
            right = of q2] {$q_3$};
        
        % Arrows
        \path [-stealth, thick]
            (q0) edge node {$a$}   (q1)
            (q0) edge [loop above]  node {$a,b$}()
            (q1) edge node {$a,b$}  (q2)
            (q2) edge node {$a,b$} (q3);
        \end{tikzpicture}
    \end{figure}
    Dans le cas du mot \textcolor{green}{$abaab$}, nous avons plusieurs exécutions possibles :
    \begin{equation*}
        q_0 \xrightarrow[]{a} q_0 \xrightarrow[]{b} q_0 \xrightarrow[]{a} q_0 \xrightarrow[]{a} q_0 \xrightarrow[]{b} q_0
    \end{equation*}
    \begin{equation*}
        q_0 \xrightarrow[]{a} q_0 \xrightarrow[]{b} q_0 \xrightarrow[]{a} q_0 \xrightarrow[]{a} q_1 \xrightarrow[]{b} q_2
    \end{equation*}
    \begin{equation*}
        q_0 \xrightarrow[]{a} q_0 \xrightarrow[]{b} q_0 \xrightarrow[]{a} q_1 \xrightarrow[]{a} q_2 \xrightarrow[]{b} q_3
    \end{equation*}
    \begin{equation*}
        \dots  
    \end{equation*}
    Comme l'exécution 3 est acceptante, le mot est accepté, effectivement, il suffit qu'une des exécutions possible soit 
    acceptante pour que le mot soit accepté. À contrario, pour le mot \textcolor{red}{$abbab$}, aucune exécution n'est
    acceptante, donc le mot n'est pas accepté.
\end{example}
\begin{definition}{Automate non-déterministe}{automate_non_déterministe}
    Un automate fini non-déterministe (AFN) $A$ sur un alphabet $\Sigma$ est un 4-uplet $(Q, q_0, F, \Delta)$ où :
    \begin{itemize}[label=\textbullet]
        \item $Q$ est un ensemble fini d'états.
        \item $q_0 \in Q$ est l'état initial.
        \item $F \subseteq Q$ est l'ensemble des états finaux.
        \item $\Delta \subseteq Q \times \Sigma \times Q$ est une relation, appelée \textbf{relation} de transition.
    \end{itemize}
\end{definition}
\begin{definition}{Langage accepté d'un automate non-déterministe}{langage_accepté_automate_non_déterministe}
    Étant donné un AFN $A$, le langage accepté par $A$, noté $L(A)$ est l'ensemble des mots pour lesquels il existe une 
    exécution acceptante de $A$.
    \begin{equation*}
        L(A) = \{w \in \Sigma^* | \exists e \text{ exécution acceptante de } A \text{ sur } w\}
    \end{equation*}
\end{definition}
\begin{remark}
    Voici deux propriétés des automates non-déterministes :
    \begin{enumerate}
        \item Pour un AFN à $n$ états, il y a au plus $n^m$ exécutions sur un mot de longueur $m$.
        \item Pour un AFN $A$ et un mot $u$, la complexité de décider si $u \in L(A)$ est polynomiale. Ceci sera démontré
        juste après.
    \end{enumerate}
\end{remark}




\subsubsection{Arbre des exécutions}
\label{subsub:arbre_des_executions}

Toutes les exécutions peuvent être représentées par un arbre, par exemple :
\begin{figure}[H]
    \centering
    \includegraphics[scale = 0.5]{pictures/arbre_executions.png}
\end{figure}
Pour voir si un mot appartient au langage, voici ce qui sera développé :
\begin{itemize}[label=\textbullet]
    \item on va exploiter l'idée précédente pour avoir une manière efficace de tester $u \in L(A)$.
    \item Soit $P \subseteq Q$ et $\sigma \in \Sigma$. On note :
    \begin{equation*}
        \text{Post}_A(P, \sigma) = \{p | \exists p' \in P \cdot (p', \sigma, p) \in \Delta\}
    \end{equation*}
    l'ensemble des états qu'on peut atteindre à partir des états de $P$ en lisant $\sigma$.
    \item L'algorithme est le suivant :\\
        TEST(A,P,u):\\
            \indent\hspace{1cm} \textbf{case} u = epsilon :\textbf{return} P $\cap$ F $\neq \emptyset$\\
            \indent\hspace{1cm} \textbf{case} u = $\sigma$v ($\sigma\in\Sigma$) : \textbf{return} TEST(A, $Post_A(P,\sigma)$, v)
    \item $u\in L(A) \Leftrightarrow TEST(A, \{q_0\}, u)$
\end{itemize}
\begin{theorem}{Appartenance au langage non-déterministe}{app_lang_nfa}
    Étant donné un AFN $A$ avec $n$ transitions et un mot $u$ de longueur $m$, on peut tester en temps $O(|u|m)$ si $u \in L(A)$.
\end{theorem}
\begin{remark}
    Effecitvement, il suffit de remarquer dans l'algorithme que calculer $Post_A(P,\sigma)$ prend $O(m)$
\end{remark}
\begin{figure}[H]
    \centering
    \includegraphics[scale = 0.4]{pictures/appartenance_langage_nfa.png}
\end{figure}
L'ensemble final est $\{q_0, q_1, q_2, q_3\}$ et il contient un mot acceptant et donc le mot est accepté. \\





\subsubsection{Test du vide}
\label{subsub:test_du_vide_nfa}
\begin{theorem}{Test du vide}{test_vide_nfa}
    Étant donné un AFN $A$ avec $n$ états et $m$ transitions, on peut tester en temps $O(n+m)$ si $L(A) = \emptyset$.
\end{theorem}
C'est la même chose que pour les automates déterministes.


\subsubsection{Déterminisme VS non-déterminisme}
\label{subsub:determinisme_vs_non_determinisme}

Tout langage accepté par un automate fini (déterministe) peut être accepté par un automate fini non-déterministe
 (et inversément).
\begin{theorem}{Déterminisme VS non-déterminisme}{determinisme_vs_non_determinisme}
    Soit $L$ un langage sur l'alphabet $\Sigma$. $L$ est accepté par un automate fini si et seulement s'il est accepté par un
    automate fini non-déterministe.
\end{theorem}
\begin{example}
    Voici un exemple et l'explication de cette conversion :
    \begin{itemize}[label=\textbullet]
        \item l'idée est la même que pour tester l'appartenance au langage :\\
        l'automate déterministe $B$ qui simule l'automate non-déterministe $A$ calcule le sous-ensemble d'états atteints.
        \item Les états de $B$ seront donc des sous-ensembles d'états de $A$.
        \item À partir d'un sous-ensemble $P \subseteq Q$, en lisant une lettre $\sigma$, $B$ va vers l'état $Post_A(P,\sigma)$.
        \item Les états acceptants de B sont les sous-ensembles de $Q$ qui contiennent un état final de $A$.
    \end{itemize}
    \begin{figure}[H]
        \centering
        \begin{tikzpicture} [node distance = 3cm, 
            on grid, 
            auto,
            every loop/.style={stealth-}]
        
        % State q0 
        \node (q0) [state, 
            initial, 
            initial text = {}] {$q_0$};
        
        % State q1    
        \node (q1) [state,
            right = of q0] {$q_1$};
        
        % State q2
        \node (q2) [state,
            right = of q1] {$q_2$};
        
        % State q3
        \node (q3) [state,
            accepting,
            right = of q2] {$q_3$};
        
        % Arrows
        \path [-stealth, thick]
            (q0) edge node {$a$}   (q1)
            (q0) edge [loop above]  node {$a,b$}()
            (q1) edge node {$a,b$}  (q2)
            (q2) edge node {$a,b$} (q3);
        \end{tikzpicture}
    \end{figure}
    \begin{figure}[H]
        \centering
        \begin{tikzpicture} [node distance = 3cm, 
            on grid, 
            auto,
            every loop/.style={stealth-}]
        
        % State q0 
        \node (q0) [state, 
            initial, 
            initial text = {}] {$q_0$};
        
        % State q1    
        \node (q1) [state,
            accepting,
            right = of q0] {$q_0,q_1$};

        % State q2
        \node (q2) [state,
            above right = of q1] {$q_0,q_1,q_2$};
        
        % State q3
        \node (q3) [state,
            below right = of q1] {$q_0,q_2$};
            
        % State q5
        \node (q5) [state,
            accepting,
            right = of q2] {$q_0,q_2,q_3$};
            
        % State q4
        \node (q4) [state,
            accepting,
            above = of q5] {$q_0,q_1,q_2,q_3$};
            
        % State q6
        \node (q6) [state,
            accepting,
            right = of q3] {$q_0,q_1,q_3$};
        
        % State q7
        \node (q7) [state,
            accepting,
            below = of q6] {$q_0,q_3$};

        % Arrows
        \path [-stealth, thick]
            (q0) edge node {$a$}   (q1)
            (q0) edge [loop above]  node {$b$}()
            (q1) edge node {$a$}  (q2)
            (q1) edge node {$b$}  (q3)
            (q2) edge node {$a$} (q4)
            (q2) edge node {$b$} (q5)
            (q3) edge node {$a$} (q6)
            (q3) edge node {$b$} (q7)
            (q4) edge [loop above] node {$a$}()
            (q4) edge node {$b$} (q5)
            (q5) edge node {$a$} (q6)
            (q5) edge [bend left] node {$b$} (q7)
            (q6) edge node {$a$} (q2)
            (q6) edge [bend left] node {$b$} (q3)
            (q7) edge [bend left] node {$b$} (q0)
            (q7) edge [bend left] node {$a$} (q1);
        \end{tikzpicture}
    \end{figure}
    L'automate $B$ construit a exponentiellement plus d'états que $A$. On peut montrer que $\forall n\geq 0$, le plus petit
    automate fini déterministe acceptant le langage $L_n$ des mots de longueur au moins $n$ dont la $n^{\text{ème}}$ lettre
    en partant de la fin est $a$ sur l'alphabet $\{a,b\}$, a $2^n$ états. Tandis que le plus petit AFN pour $L_n$ a $O(n)$ 
    états.
\end{example}


\subsection{Expressions rationnelles}
\label{sub:expressions_rationnelles}

\begin{definition}{Expression rationnelle}{expression_rationnelle}
    Une expression rationnelle $E$ sur un alphabet $\Sigma$ est une expresssion qui respecte la grammaire suivante :
    \begin{equation*}
        E ::= \epsilon\; |\; a\; |\; \emptyset\; |\; (E + E)\; |\; (E \cdot E)\; |\; (E^*)
    \end{equation*}
    pour tout $a \in \Sigma$.
\end{definition}
\begin{definition}{Opérations sur les langages}{op_langage}
    Soient $L,L_1,L_2\subseteq\Sigma^*$ trois langages. Alors :
    \begin{itemize}[label=\textbullet]
        \item $L_1\cdot L_2 = \{u_1 u_2 | u_1\in L_1 \wedge u_2\in L_2\}$ (noté aussi $L_1L_2$)
        \item $L^* = \{u_1,\dots,u_k | k\geq 0, u_i\in L \forall i\in \{1,\dots,k\}\}$
    \end{itemize}
\end{definition}
\begin{example}
    Voici deux exemples :
    \begin{itemize}[label=\textbullet]
        \item Si $L_1 = \{a,b\}$ et $L_2 = \{a,bb\}$ alors $L_1\cdot L_2 = \{aa,abb,ba,bbb\}$
        \item Si $L = \{a\}$ alors $L^* = \{a^n|n\geq 0\}$
    \end{itemize}
\end{example}
\begin{definition}{Sémantique des expresssions rationnelles}{sem_rati}
    La sémantique d'une expression rationnelle $E$ sur $\Sigma$ est donnée par un langage, noté $L(E)$, défini inductivement
    par :
    \begin{itemize}
        \item $L(\epsilon) = \{\epsilon\}$
        \item $L(a) = \{a\}$ pour tout $a\in\Sigma$
        \item $L(\emptyset) = \emptyset$
        \item $L(E_1 + E_2) = L(E_1) \cup L(E_2)$
        \item $L(E_1 \cdot E_2) = L(E_1) \cdot L(E_2)$
        \item $L(E^*) = L(E)^*$
    \end{itemize}
\end{definition}
\begin{example}
    Sur $\Sigma = \{a,b\}$, voici quelques exemples :
    \begin{itemize}
        \item $L((a+b)^*a(a+b)*)$ est l'ensemble des mots qui contiennent au moins un $a$.
        \item $L(a^*b^*)$ est l'ensemble des mots qui sont des séquences de $a$ suivies des séquences de $b$.
    \end{itemize}
\end{example}



\subsubsection{Expressions vers automates}
\label{subsub:expressions_vers_automates}

\begin{theorem}{Théorème de Kleene}{th_kleene}
    Tout langage est reconnaissable par un automate si et seulement s'il est définissable par une expression rationnelle.
\end{theorem}
La construction se fait par induction sur les expressions. Pour toute expression $E$, on va construire un AFN $A_E$ tel que
$L(A_E) = L(E)$ (il suffit ensuite de déterminiser $A_E$ avec la construction des sous-ensembles pour terminer la preuve).
\begin{itemize}[label=\textbullet]
    \item si $E = \epsilon$, alors $A_E = (\{q_0\}, q_0, \{q_0\}, \Delta := \emptyset)$
    \item si $E = a$, avec $a\in\Sigma$, alors $A_E = (\{q_0,q_1\}, q_0, \{q_1\}, \Delta := \{(q_0,a,q_1)\})$
    \item si $E = \emptyset$, alors $A_E = (\{q_0\}, q_0, \emptyset, \Delta := \emptyset)$
\end{itemize}


\subsubsubsection{$E=F+G$}
On construit par induction $A_F$ et $A_G$, à partir desquels on construit $A_E$ tel que :
\begin{equation*}
    L(A_E) = L(A_F) \cup L(A_G)
\end{equation*}
C'est la clôture par union (cf. Théorème \ref{clôture_union_intersection}) seulement, grâce au non-déterminisme, on peut 
faire plus simple :
\begin{itemize}[label=\textbullet]
    \item en lisant la première lettre $\sigma$, on va soit dans le premier automate soit dans le deuxième, en utilisant le
    non-déterminisme.
    \item précisément, $A_E$ a un état initial $q_0$. On note $q_0^F$ l'état initial de $A_F$ et $q_0^G$ l'état initial de
    $A_G$. Pour toute lettre $\sigma$, pour toute transition $(q_0^F,\sigma,q)\in\Delta_F$, on ajoute la transition $(q_0,
    \sigma, q)$ à $\Delta_E$. De même pour $A_G$.
    \item On rend le nouvel état initial acceptant si $q_0^F$ ou $q_0^G$ est acceptant. (pour accepter le mot vide)
\end{itemize}
\begin{figure}[H]
    \centering
    \includegraphics[scale = 0.4]{pictures/EF+G.png}
\end{figure}


\subsubsubsection{$E=FG$}
On construit $A_F$ et $A_G$ par induction, puis pour toute lettre $\sigma$, pour tout état de $A_F$ qui allait vers un état
final de $A_F$ en lisant $\sigma$, on ajoute une transition vers l'état initial de $A_G$. 
\warningbox{Les états acceptants de $A_F$ ne sont plus acceptants dans $A_E$ et l'état inital de $A_E$ est l'état initial de
$A_F$.}
\begin{figure}[H]
    \centering
    \includegraphics[scale = 0.2]{pictures/AFG.jpeg}
\end{figure}


\subsubsubsection{$E=G^*$}
Il faut faire attention au fait que $\epsilon\in L(E)$. On réécrit d'abord $E$ comme $E = \epsilon +G^+$ où $G^+$ est 
l'ensemble des mots de la forme $u_1u_2\dots u_k$ avec $k\geq 1$ et $u_i\in L(G)$ pour tout $i$.\\
Ensuite, on constrit $A_\epsilon$ et $A_G$ par induction. On montre maintenant comment obtenir $A_{G^+}$ et enfin $A_E$
sera obtenu en utilisant la clôture par union de $A_\epsilon$ et $A_{G^+}$. Pour construire $A_{G^+}$, c'est similaire à
la concaténation sauf qu'on ajoute des transitions vers l'état initial de $G$ :
\begin{figure}[H]
    \centering
    \includegraphics[scale=0.2]{pictures/AG+.jpeg}
\end{figure}


\subsection{Minimisation}
\label{sub:minimisation}

Étant donné un automate $A$, l'objectif de la minimisation est de construire un automate complet $M$ tel que $M$ a un nombre
d'états minimal et $M$ est équivalent à $A$. (cf. Définition \ref{automates_équivalents})
\begin{example}
    Voici un exemple d'un automate et de son automate minimal équivalent :
    \begin{figure}[H]
        \centering
        \begin{tikzpicture} [node distance = 3cm, 
            on grid, 
            auto,
            every loop/.style={stealth-}]
        
        % State q0 
        \node (q0) [state, 
            initial, 
            accepting, 
            initial text = {}] {$q_0$};
        
        % State q1    
        \node (q1) [state,
            accepting,
            right = of q0] {$q_1$};
        
        % Arrows
        \path [-stealth, thick]
            (q0) edge[bend left] node {$a$}   (q1)
            (q1) edge[bend left] node {$a$}   (q0)
            (q0) edge [loop above]  node {b}()
            (q1) edge [loop above]  node {b}();
        \end{tikzpicture}
    \end{figure}

    \begin{figure}[H]
        \centering
        \begin{tikzpicture} [node distance = 3cm, 
            on grid, 
            auto,
            every loop/.style={stealth-}]
        
        % State q0 
        \node (q0) [state, 
            initial, 
            accepting, 
            initial text = {}] {$q_0$};
        
        % Arrows
        \path [-stealth, thick]
            (q0) edge [loop above]  node {a,b}();
        \end{tikzpicture}
    \end{figure}
\end{example}
\begin{definition}{Sémantique}{sémantique_minimisation}
    Soit $A=(Q,q_0,F,\delta)$ un automate complet sur un alphabet $\Sigma$. Pour tout mot $u\in\Sigma^*$, pour tout état $q\in
    Q$, on note :
    \begin{itemize}[label=\textbullet]
        \item $q\cdot u$ l'état atteint à partir de $q$ en lisant $u$. (il existe car $A$ est complet)
        \item $L_q$ le langage formé des mots $u$ tels que $q\cdot u\in F$.
        \begin{equation*}
            L_q = \{u\in\Sigma^* | q\cdot u\in F\}
        \end{equation*}
        est le langage des mots acceptés à partir de $q$.
        \item pour tout $p,q\in Q, p\equiv_A q$ si $L_p = L_q$.
    \end{itemize}
\end{definition}
\begin{remark}
    $L_{q_0} = L$, $\epsilon\in L_q$ pour tout $q\in F$. $\equiv_A$ est une relation d'équivalence, on note $[p]_A$ la classe
    de tout état $p$.
\end{remark}
\noindent À partir d'un automate $A$, on calcule un automate $M_A$ comme suit :
\begin{enumerate}
    \item Il suffit de calculer toutes les classes d'équivalence de $Q$ pour $\equiv_A$, ce qui donnera les états.
    \item La classe de $q_0$ est l'état initial.
    \item Toute classe qui contient un état final est finale.
    \item On met une transition de l'état $[q]\equiv_A$ à l'état $[q\cdot a]_{\equiv_A}$ en lisant $a$, pour tout $a\in\Sigma$.
    et tout $q\in Q$.
\end{enumerate}
\begin{theorem}{Automate minimal}{automate_minimal}
    L'automate minimal $M_A$ est minimal en nombre d'états et il est l'unique automate minimal complet qui accepte $L(A)$.
\end{theorem}
\begin{example}
    Voici un exemple de calcul de l'automate minimal :
    \begin{figure}[H]
        \centering
        \includegraphics[scale=0.3]{pictures/automate_minimal.jpeg}
    \end{figure}
\end{example}
En pratique, ce n'est pas très efficace, on peut faire mieux, sans passer par le test d'équivalence et par raffinement 
successif de relations d'équivalence qui convergent vers $\equiv_A$. Cela est même possible en $O(n\cdot log_2(n))$ où
$n=|Q|$. Pour les AFNs, le problème d'optimisation est plus compliqué, décider si un AFN est équivalent à un AFN avec au 
plus $k$ états est PSPACE-dur.
\newpage

\section{Introduction à la théorie de la complexité}
\label{sec:complexité}
\subsection{Problèmes de décision}
\label{sub:problemes_de_decision}

\begin{definition}{Problème de décision}{problème_de_décision}
  Un problème de décision est un langage $P\subseteq\Sigma^*$.
\end{definition}
\begin{remark}
    Chaque langage $P$ représente un problème dont la réponse est oui ou non, en l'identifiant à sa fonction caractéristique
    $\chi_P$ :
    \begin{equation*}
        \begin{aligned}
            \chi_P\; :\; \Sigma^* &\rightarrow \{0,1\} \\
            u &\mapsto \begin{cases}
                1 & \text{si } u \in P \\
                0 & \text{sinon.}
            \end{cases}
        \end{aligned}
    \end{equation*}
    Étant donné un mot $u\in \Sigma^*$, il faut décider si $u\in P$ ou non.
\end{remark}
\begin{example}
    Avec $\Sigma = \{0,1\}$,
    \begin{equation*}
        \text{PRIME} = \{10,11,101,111,cdots\} 
    \end{equation*}
    l'ensemble des nombres premiers en binaire.
\end{example}
Il est parfois compliqué de représenter un problème comme un langage, effectivement, comment faire pour :
\begin{itemize}[label=\textbullet]
    \item ENTRÉE : un graphe $G$ non dirigé et un entier $k\in\mathbb{N}$.
    \item SORTIE : $1$ ssi on peut colorier $G$ avec $k$ couleurs.
\end{itemize}
On pourrait représenter ceci avec l'alphabet $\Sigma = \{0,1,\#,\$$\} avec ($i,j$) chaque arête qui pourrait être codée par 
le mot $\overline{i}\#\overline{j}$ où $\overline{i}$ est le codage binaire du sommet $i$. La paire ($G,k$) pourrait se 
coder par le mot :
\begin{equation*}
    \overline{i_1}\#\overline{j_1}\$\overline{i_2}\#\overline{j_2}\$\cdots\$\overline{i_m}\#\overline{j_m}\$\overline{k}
\end{equation*}
\warningbox{Le codage peut influencer la complexité. Il sera parfois nécessaire de changer le codage pour obtenir une complexité 
plus faible.}

\subsection{Problème d'optimisation}
\label{sub:probleme_d_optimisation}
\begin{itemize}[label=\textbullet]
    \item Un problème d'optimisation est un problème où l'on souhaite maximiser/minimiser une certaine quantité.
    \begin{itemize}[label=$\rightarrow$]
        \item Trouver la longueur d'un plus court chemin entre deux sommets ($s$ et $t$) d'un graphe.
    \end{itemize}
    \item Un problème d'optimisation peut être associé à un problème de décision.
    \begin{itemize}[label=$\rightarrow$]
        \item Existe-t-il un chemin de longueur au plus $k$ de $s$ à $t$ ?
    \end{itemize}
    \item Si on sait résoudre le problème de décision, on peut parfois résoudre le problème d'optimisation.
    \begin{itemize}[label=$\rightarrow$]
        \item Pour trouver le plus court chemin de $s$ à $t$ dans un graphe à $n$ sommets, on peut faire une recherche
        dichotomique.
    \end{itemize}
\end{itemize}

\subsection{Algorithme de décision}
\label{sub:algorithme_de_decision}
\begin{definition}{Algorithme de décision}{algorithme_de_décision}
    Un problème $P\subseteq\Sigma^*$ est décidé par un algorithme $A$ si pour tout mot $u\in\Sigma^*$;
    \begin{itemize}[label=\textbullet]
        \item $A$ se termine et retourne 1 si $u\in P$.
        \item $A$ se termine et retourne 0 si $u\notin P$.
    \end{itemize}
    Un problème est décidable s'il existe un algorithme qui le décide.
\end{definition}

\subsection{La classe $\mathcal{P}$}
\label{sub:la_classe_p}
\begin{definition}{Classe $\mathcal{P}$}{classe_p}
    La classe $\mathcal{P}$ est la classe des problèmes pouvant être décidés en temps polynomial. Plus précisément,
    un problème $P\subseteq\Sigma^*$ est dans $\mathcal{P}$ s'il existe un algorithme $A$ et une constante $k$ tel que
    pour tout mot $u$ de longueur $n$,
    \begin{itemize}[label=\textbullet]
        \item $A$ retourne 1 en temps $O(n^k)$ si $u\in P$.
        \item $A$ retourne 0 en temps $O(n^k)$ si $u\notin P$.
    \end{itemize}
\end{definition}
\begin{example}
    Exemples de problèmes dans $\mathcal{P}$ :
    \begin{itemize}[label=\textbullet]
        \item décider si un tableau est trié.
        \item décider si un entier codé en unaire est premier (facile)
        \item décider si un entier codé en binaire est premier (difficile)
    \end{itemize}
\end{example}


\subsection{Algorithme de vérification}
\label{sub:algorithme_de_verification}
\begin{definition}{Algorithme de vérification}{algorithme_de_vérification}
    Un algorithme de vérification pour un problème $P\subseteq\Sigma^*$ est un algorithme de décision (retournant 1 ou 0)
    $A$ prenant deux mots en argument, qui termine pour toute entrée, tel que:
    \begin{equation*}
        P = \{u\in\Sigma^* | \exists v \in \Sigma^*, A(u,v)=1\}
    \end{equation*}
    Lorsque $A(u,v)=1$, $v$ est appelé un certificat pour $u$.
\end{definition}

\subsection{La classe $\mathcal{NP}$}
\label{sub:la_classe_np}
\begin{definition}{Classe $\mathcal{NP}$}{classe_np}
    Un problème $P$ est dans $\mathcal{NP}$ s'il existe un algorithme de vérification $A$ de complexité polynomiale en temps
    et une constante $k$, tels que pour toute entrée $u$, les deux afformations suivantes sont équivalentes:
    \begin{enumerate}
        \item $u\in P$
        \item il existe un certificat $v$ de longueur polynomiale dans $u$ (i.e. $|v| = 0(|u|^k$)) tel que $A(u,v)=1$.
    \end{enumerate}
\end{definition}
\begin{example}
    Exemples de problèmes dans $\mathcal{NP}$ :
    \begin{itemize}[label=\textbullet]
        \item décider qu'un ensemble de clauses est satisfaisable.
        \item voyageur de commerce : étant donné $n$ villes, les distances entre les villes et un entier $d$, on voudrait
        savoir s'il existe un cycle de longueur $\leq d$ passant par toutes les villes une et une seule fois
        \item coloriage de graphes : peut-on choisiir un graphe avec moins de $k$ couleurs sans que deux sommets aient la 
        même couleur ?
        \item tous les problèmes de la classe $\mathcal{P}$ sont dans $\mathcal{NP}$.
    \end{itemize}
\end{example}
Tout problème $\mathcal{NP}$ peut être décidé par un algorithme de complexité exponentielle en temps car étant donné un mot
$u$ en entrée, il suffit d'énumérer tous les certificats de longueur au plus $a\cdot|u|^k$ et d'appeler l'algorithme de 
vérification. Pour SAT, par exemple, il suffit d'énumérer toutes les interprétations possibles et les tester.\\
Il existe une fameuse conjecture en théorie de la complexité qui est la suivante : 
\begin{equation*}
    \mathcal{P}\neq\mathcal{NP}
\end{equation*}

\subsection{Temps non-déterministe polynomial}
\label{sub:temps_non_deterministe_polynomial}
La classe $\mathcal{NP}$ peut également être définie comme la classe des problèmes pouvant être décidés en temps polynomial
par un algorithme non-déterministe. Si la réponse au problème est oui, alors il existe une exécution de l'algorithme après 
laquelle l'algorithme répondra oui.
\begin{example}
    SAT, choisir aléatoirement une valeur de vérité pour chaque variable et vérifier en temps linéaire qu'elles satisfont
    la formule.
\end{example}

\subsection{Réductions, $\mathcal{NP}$-dureté et $\mathcal{NP}$-complétude}
\label{sub:reductions_np_durete_et_np_completude}
\begin{definition}{$\mathcal{NP}$-dur}{np-dur}
    Un problème de décision $P$ est $\mathcal{NP-dur}$ si pour tout problème $P'$ de $\mathcal{NP}$ se réduit à $P$ en temps
    polynomial, i.e. qu'il existe un algorithme $T$ de complexité polynomiale en temps, qui transforme tout mot $u'$ en un 
    mot $T(u')$ tel que $u'\in P'$ ssi $T(u')\in P$.
\end{definition}
Un problème qui est dans $\mathcal{NP}$ et $\mathcal{NP-dur}$ est dit $\mathcal{NP}$-complet, ce sont les plus compliqués de
la classe $\mathcal{NP}$.
\warningbox{Un problème $P$ peut être dans $\mathcal{NP}$-dur sans être dans $\mathcal{NP}$.}

Pour démontrer que $P$ est dans $\mathcal{NP}$-complet, il faut:
\begin{enumerate}
    \item Démontrer qu'il est dans $\mathcal{NP}$.
    \item Démontrer qu'il est dans $\mathcal{NP}$-dur. Pour démontrer qu'un problème $P$ est $\mathcal{NP}$-dur, il faut
    partir d'un problème connu comme étant $\mathcal{NP}$-dur, qu'on réduit dans notre problème en temps polynomial. Le 
    premier problème $\mathcal{NP}$-dur est le \textbf{Théorème de Cook}.
\end{enumerate}
\begin{figure}[H]
    \centering
    \includegraphics[scale=0.3]{pictures/NP.png}
    \caption{Relations entre les classes de complexité}
\end{figure}
\begin{theorem}{}{} 
    Si $\mathcal{P}\neq\mathcal{NP}$ et un problème $A$ est $\mathcal{NP}$-complet, alors $A\notin \mathcal{P}$.
\end{theorem}
\begin{proof}
    On va supposer que $A\in\mathcal{P}$ et en déduire la contradiction $\mathcal{P}=\mathcal{NP}$. On sait que $\mathcal{P}
    \subseteq\mathcal{NP}$, il nous reste à démontrer que $\mathcal{NP}\subseteq\mathcal{P}$. Soit $B$ un problème de 
    $\mathcal{NP}$ quelconque, montrons qu'il est dans $\mathcal{P}$. \\
    Comme $A$ est $\mathcal{NP}$-complet, $B$ se réduit à $A$ en temps polynomial. Comme $A\in\mathcal{P}$, nous avons un 
    algorithme en temps polynomial pour résoudre $B$ :
    \begin{itemize}[label=\textbullet]
        \item ENTRÉE : Instance $I$ de $B$
        \begin{enumerate}
            \item Transformer $I$ en une instance $I'$ de $A$ en temps polynomial. \textcolor{green}{$O(n^c)$}
            \item Résoudre $I'$ en temps polynomial. \textcolor{green}{$O(n^d)$}
        \end{enumerate}
    \end{itemize}
    Supposons que la réduction se fasse en temps $O(n^c)$ pour une constante $c$ et que $A$ se résout en temps $O(n^d)$ pour
    une constante $d$. Alors $B$ se résout en temps $O(n^{cd})$ :
    \begin{enumerate}
        \item Puisque l'étape 1 se fait en temps $O(n^c)$, où $n$ est la taille de $I$, la taille de $I'$ est en $O(n^c)$.
        \item L'étape 2 est appliquée sur une entrée de taille $O(n^c)$, donc elle prend un temps $O((n^c)^d) = O(n^{cd})$.
    \end{enumerate}
\end{proof}
\begin{lemma}{}{}
    Si vous prouvez qu'un problème $A$ est dans $\mathcal{P}$ et est $\mathcal{NP}$-complet, soit vous avez démontré 
    $\mathcal{P}=\mathcal{NP}$, soit vous avez fait une erreur.
\end{lemma}
\begin{lemma}{}{}
    Si vous ne trouvez pas d'algorithme en temps polynomial pour votre problème, alors peut-être qu'il est $\mathcal{NP}$-complet.
    et dans ce cas vous avez peu de chances d'en trouver un.
\end{lemma}

\subsection{Réduction de SAT vers 3-SAT}
\newpage

\section{Logique des prédicats}
\label{sec:logique_prédicats}
\subsection{Introduction}

En logique des prédicats ;
\begin{itemize}[label=\textbullet]
    \item on ajoute les quantificateurs.
    \item on généralise les valeurs que peuvent prendre les variables.
    \item on ajoute des relations poiur décrire certianes relation entre ces valeurs.
    \item on ajoute des symboles de fonctions à la syntaxe.
\end{itemize}
\begin{example}
    $\forall x\forall y \cdot \text{PremierEntreEux}(x,y)\leftrightarrow \exists x'\exists y' \cdot x.x'
    + y.y' = 1$
    \begin{itemize}[label=\textbullet]
        \item PremierEntreEux est un prédicat à deux arguments.
        \item 1 est appelée constante.
        \item $x x' + y y'$ est un terme formé avec les fonctions $\times$ et $.$
    \end{itemize}
\end{example}

\subsection{Syntaxe}
\label{subsec:syntaxe}
\subsubsection{Alphabet}
\label{subsubsec:alphabet}
L'alphabet d'un langage du premier ordre comporte d'abord les symboles suivants qui sont communs à tous ces langages :
\begin{itemize}[label=\textbullet]
    \item les connecteurs : $\neg,\wedge,\vee,\rightarrow\leftrightarrow$;
    \item les paranthèses : $(,)$;
    \item le quantificateur universel $\forall$ et le quantificateur existentiel $\exists$;
    \item un ensemble infini $\vee$ de symboles de variables $x,y,z,\dots$;
\end{itemize}
\begin{definition}{Langage de la logique du premier ordre}{lg_1_ordre}
    Un langgage $\mathcal{L}$ de la logique du premier ordre est caractérisé par :
    \begin{itemize}[label=\textbullet]
        \item des symboles de relations (prédicats), notés $p,q,r,s,\dots$;
        \item des symboles de fonctions, notés $f,g,h,\dots$;
        \item des symboles de constantes, notés $c,d,e,\dots$;
    \end{itemize}
    À chaque prédicat $p$, respectivement fonction $f$, on associe un entier strictement positif appelé 
    l'arité de $p$, respectivement de $f$, càd le nombre d'arguments de $p$, respectivement $f$. On notera parfois $p|_n$ 
    et $f|_n$ pour indiquer que $p$ (respectivement $f$) est un symbole de relation (respectivement de fonction) d'arité $n$.\\
    Le prédicat "=" sera toujours présent.
\end{definition}
\begin{example}
    Exemples de langages :
    \begin{itemize}[label=\textbullet]
        \item $\mathcal{L}_1 = \{r|_1,c\}$ contient un prédicat unaire $r$ et une constante $c$;
        \item $\mathcal{L}_2 = \{r|_2,f|_1,g|_2,h|_2,c,d\}$ contient un prédicat binaire $r$, une fonction unaire $f$,
        deux symboles de fonctions binaires $g$ et $h$ et deux constantes $c$ et $d$.\\
    \end{itemize}
\end{example}

\subsubsection{Construction des termes}
\label{subsubsec:construction_termes}
\begin{definition}{Termes d'un langage}{termes_langage}
    L'ensemble des termes d'un langage $\mathcal{L}$ est le plus petit ensemble qui contient les symboles de constantes et de
    variables et qui est clos par application des fonction.\\
    L'ensemble des termes, noté $\mathcal{T}$, est le plus petit ensemble satisfaisant :
    \begin{enumerate}
        \item Tout symbole de constante ou variable est un terme.
        \item Si $f$ est un symbole de fonction d'arité $n$ et $t_1,t_2,\cdots,t_n$ sont des termes alors $f(t_1,t_2,\cdots,t_n)$ est un terme.
        est un terme.
    \end{enumerate}
\end{definition}
\begin{example}
    Voici des exemples :
    \begin{itemize}[label=\textbullet]
        \item Les seuls termes du langages $\mathcal{L}_1$ sont la constante $c$ et les variables.
        \item Les expressions suivantes sont des termes du langage $\mathcal{L}_2$ : 
        \begin{itemize}[label=\textbullet]
            \item $f(c)$
            \item $f(h(f(c),d))$
            \item $f(y)$
            \item $f(h(f(x),f(d)))$
        \end{itemize}
    \end{itemize}
\end{example}
\noindent Un terme est \textbf{clos} s'il est sans variable. Ici, $f(c)$ est clos.

\subsubsection{Construction des formules}
\label{subsubsec:construction_formules}
\begin{definition}{Formules d'un langage}{formules_langage}
    L'ensemble des formules du langage $\mathcal{L}$, que l'on désigne par $\mathcal{F(L)}$, est défini par la grammaire suivante:
    \begin{equation*}
        \phi ::= p(t_1,\cdots,t_n)|\phi\wedge\phi|\phi\vee\phi|\neg\phi|\phi\rightarrow\phi|\phi\leftrightarrow\phi|\exists x
        \cdot\phi|\forall x\cdot\phi|(\phi)
    \end{equation*}
    \begin{itemize}[label=\textbullet]
        \item $t_1,\cdots,t_n$ sont des termes;
        \item $p$ est un symbole de relation;
        \item $\exists x$ est le quantificateur existentiel;
        \item $\forall x$ est le quantificateur universel;
    \end{itemize}
\end{definition}
\begin{example}
    Voici des exemples :
    \begin{itemize}[label=\textbullet]
        \item La formule $r(c)\vee\neg\exists x\cdot r(x)$ est une formule du langage $\mathcal{L}_1$.
        \item Exemples de formules du langage $\mathcal{L}_2$ :
        \begin{itemize}[label=\textbullet]
            \item $\forall x\cdot\exists y(g(x,y) = c\wedge g(x,y) = c)$
            \item $\forall x\cdot\neg(f(x)=c)$
        \end{itemize}
    \end{itemize}
\end{example}

\subsubsection{Règles de précédence}
\label{subsubsec:règles_précédence}
Pour les Booléens, ce sont les mêmes règles de précédence que dans la logique propositionnelle (cf. \ref{sec:logique_propositionnelle}).
Les quantificateurs ont la même priorité que la négation.
\begin{equation*}
    \forall x\cdot\neg p(x,y)\vee p(y,x) \equiv (\forall x\cdot\neg(p(x,y)))\vee p(y,x)
\end{equation*}

\subsubsection{Variables libres et liées}
\label{subsubsec:variables_libres_liées}
\begin{definition}{Occurence de variable}{occurence_variable}
    Une occurence d'une variable dans une formule est un couple constitué de cette variable et d'une place effective, càd qui
    ne suit pas un quantificateur. 
\end{definition}
\begin{example}
    Dans la formule :
    \begin{equation*}
        r(x,z)\rightarrow\forall z\cdot (r(y,z)\vee y=z)
    \end{equation*}
    La varible $x$ possède une occurence, la variable $y$ deux et la variable $z$ trois.
\end{example}
\begin{definition}{Variables libres ou liées}{variables_libres_liées}
    \begin{itemize}[label=\textbullet]
        \item Une occurence d'une variable $x$ dans une formule $\phi$ est une occurence \textbf{libre} si elle ne se trouve 
        dans aucune sous-formule de $\phi$, qui commence par une quantification $\forall x$ ou $\exists x$.
        \item Dans le cas contraire, l'occurence est dite \textbf{liée}.
        \item Une variable est libre dans une formule si elle a au moins une occurence libre dans cette formule.
        \item Une formule est close est une formule sans variable libre.
        \item On note $\text{Libres}(\phi)$ l'ensemble des variables libres de $\phi$.
    \end{itemize}
\end{definition}
\begin{example}
    Dans $\exists x\cdot p(x,y)$, l'occurence de $x$ est liée et celle de $y$ est libre.
\end{example}



\subsection{Sémantique}
\label{subsec:sémantique}

\subsubsection{Interprétation des formules}
\label{subsubsec:interprétation_formules}

\begin{definition}{Structure}{structure}
    Une structure $\mathcal{M}$ pour un langage $\mathcal{L}$ se compose d'un ensemble non vide $M$ appelé le domaine et d'une 
    interprétation des symboles de prédicats par des relations sur $M$, des symboles de fonctions par des fonctions de $M$ et 
    des constantes par des éléments de $M$.\\
    Plus précisément, une structure est composée de :
    \begin{itemize}[label=\textbullet]
        \item d'un sous-ensemble de $M^n$, noté $p^\mathcal{M}$, pour chaque symbole de prédicat $r$ d'arité $n$ dans $\mathcal{L}$;
        \item d'une fonction \textbf{totale} de $M^m$ dans $M$, notée $f^\mathcal{M}$, pour chaque symbole de fonction $f$ d'arité 
        $m$ dans $\mathcal{L}$;
        \item d'un élément de $M$, noté $c^\mathcal{M}$, pour chaque symbole de constante $c$ dans $\mathcal{L}$.
    \end{itemize}
\end{definition}
\begin{example}
    \begin{itemize}
        \item Pour le langage $\mathcal{L}_1 = (r|_1c)$, la structure $\mathcal{M}_1 = (\mathbb{N}, r^{\mathcal{M}_1}, c^{\mathcal{m}_1})$ 
        avec $r^{\mathcal{M}_1}$ l'ensemble des nombres premiers et $c^{\mathcal{M}_1} = 2$ est une interprétation de $\mathcal{L}_1$.
        \item Pour le langage $\mathcal{L}_2 = (r|_2,f|_1,g|_2,h|_2,c,d)$, on peut prendre la structure sur les réels :
        \begin{equation*}
            \mathcal{M}_2 = (\mathbb{R}, \leq, +1,+,\times,0,1)
        \end{equation*}
        avec la fonction $+1$ qui à $x$ associe $x+1$.
    \end{itemize}
\end{example}
\begin{theorem}{Formule satisfaite}{formule_satisfaite}
    Une formule $\phi$ construite sur un langage $\mathcal{L}$ est satisfaite dans une structure $\mathcal{M}$ et pour une valuation
    $v$ donnanat une valeur aux variables de l'ensemble $\mathcal{V}$ (noté $\mathcal{M}, v\vDash\emptyset$) ssi :
    \begin{itemize}[label=\textbullet]
        \item si $\phi\equiv r(t_1,t_2,\cdots,t_n)$ et $t_i^{\mathcal{M},v}=b_i$ pour $i=1,\cdots,n$, alors $\phi$ est vraie
        ssi $(b_1,b_2,\cdots,b_n)\in r^{\mathcal{M}}$;
        \item si $\phi\equiv\neg\psi_1, \phi\equiv\psi_1\vee\psi_2,\phi\equiv\psi_1\wedge\psi2,\phi\equiv\psi_1\rightarrow\psi_2,
        \phi\equiv\psi_1\leftrightarrow\psi_2$ alors la valeur de $\phi$ est calculée àpd valeurs de $psi_1$ et $psi_2$ comme
        dans le cas propositionnel.
        \item si $\phi\equiv\exists x\cdot\psi$, alors $\phi$ est vraie ssi \textbf{il existe} une valuation $v'$ telle que $\mathcal{M},v'\vDash
        \psi$ et $v'$ est d'accord (=$v'(x)=v(x)$) avec $v$ sur Libres($phi$).
        \item si $\phi\equiv\forall x\cdot\psi$, alors $\phi$ est vraie ssi \textbf{pour toute} valuation $v'$ qui est d'accord avec $v$ sur
        Libres($\phi$), on a $\mathcal{M},v'\vDash\psi$.
    \end{itemize}
\end{theorem}
Lorsque $\mathcal{M},v\vDash\emptyset$, on dit que $\mathcal{M},v$ satisfait $\phi$ ou encore que ($\mathcal{M},v$) est un modèle de $\phi$.
De pus, lorsque $\phi$ est une formule close, alors sa valeur de vérité dans un couple ($\mathcal{M},v$), ne dépend pas de $v$.
On omettera de mentionner $v$ dans ce cas.
\begin{example}
    \begin{itemize}[label=\textbullet]
        \item Prenons $\mathcal{L}_1=\{r|_2,c\}$. La formule close suivante :
        \begin{equation*}
            \begin{aligned}
                &\forall x\cdot r(x,x)\\
                \wedge&\forall x\cdot\forall y\cdot(r(x,y)\rightarrow r(y,x))\\
                \wedge&\forall x\cdot\forall y\cdot\forall z\cdot(r(x,y)\wedge r(y,z)\rightarrow r(x,z))
            \end{aligned}
        \end{equation*}
        exprime qu'une structure $(D,R,a)$ est un modèle de la formule ssi $R$ est une relation d'équivalence.
        \item est-ce que $\exists x\cdot \forall y\cdot r(x,y)$ est vraie dans ($\mathbb{N},\leq$)?\\
        "Est-ce qu'il existe un entier $x\in\mathbb{N}$ tel que $\forall y\in\mathbb{N}$, $x\leq y$?" oui, $x=0$.
        
    \end{itemize}
\end{example}

\subsubsection{Interprétation des termes dans une structure}
\label{subsubsec:interprétation_termes_structure}

\begin{definition}{Valuation}{valuation}
    Etant donnés un ensemble de variables $\mathcal{V}$ et un domaine $M$, une \textit{valuation} pour les variables de $\mathcal{V}$
    dans $M$ est une fonction $v:\mathcal{V}\rightarrow M$ qui attribue à chaque variable $x\in \mathcal{V}$, une valeur $v(x)\in M$.
\end{definition}
\begin{definition}{Interprétation de termes}{inter_terme}
    L'interprétation d'un terme $t$ (dont les variables sont dans $\mathcal{V}$) dans une structure de domaine $M$ et selon une 
    valuation $v$ est un élément $t^{\mathcal{M},v}\in M$, inductivement défini de la façon suivante :
    \begin{itemize}[label=\textbullet]
        \item si $t=c$ alors $t^{\mathcal{M},v} = c^{\mathcal{M}}$;
        \item si $t=x$ alors $t^{\mathcal{M},v} = v(x)$ est $v(x)$;
        \item si $t=f(t_1,\cdots,t_n)$ alors $t^{\mathcal{M},v} = f^{\mathcal{M}}(t_1^{\mathcal{M},v},\cdots,t_n^{\mathcal{M},v})$.
    \end{itemize}
\end{definition}
\begin{example}
    Soit $\mathcal{L}_2 = (r|_2,f|_1,g|_2,h|_2,c,d)$ et $\mathcal{M}_3 = (\mathbb{N},\leq,+1,+,\times,0,1)$.\\
    L'interprétation dans $\mathcal{M}_3$ du terme 
    \begin{equation*}
        t_1 \equiv g(y,h(c,x))
    \end{equation*}
    selon la valuation $v$ telle que $v(x)=3, v(y)=4,v(z)=6$ est :
    \begin{equation*}
        t_1^{\mathcal{M}_3,v} = 4 + (0\times 3) = 4
    \end{equation*}
    L'interprétation du terme 
    \begin{equation*}
        t_2\equiv f(g(d,h(y,z)))
    \end{equation*}
    est :
    \begin{equation*}
        t_2^{\mathcal{M}_3,v} = 1 + (4\times 6) + 1 = 26
    \end{equation*}
\end{example}

\subsubsection{Formules satisfaisables, valides et équivalentes}
\label{subsubsec:formules_satisfaisables_valides_équivalentes}
\begin{enumerate}
    \item Une formule $\phi$ close est \textbf{satisfaisable} si elle un modèle. (il n'existe pas d'algo pour vérifier la satisfaisabilité
    d'une formule).
    \item Une formule $\phi$ close est \textbf{valide} si toutes les structures sont des modèles de $\phi$.
    \item Deux formules $\phi_1,\phi_2$ telles que Libres($\phi_1$) = Libres($\phi_2$) sont \textbf{équivalentes} si la formule
    $\forall x_1\cdots\forall x_n(\phi_1\leftrightarrow\phi_2)$ est valide, avec $\{x_1,\cdots,x_n\}$=Libres($\phi_1$).
    \begin{example}
        Voici quelques exemples de formules équivalentes :
        \begin{itemize}[label=\textbullet]
            \item $\forall x\cdot(\phi\wedge\psi)$ et $(\forall x\cdot\phi)\wedge(\forall x\cdot\psi)$
            \item $\exists x\cdot(\phi\vee\psi)$ et $(\exists x\cdot\phi)\vee(\exists x\cdot\psi)$
        \end{itemize}
    \end{example}
\end{enumerate}

\subsubsection{Exemples de traductions de texte vers formules}
\label{subsubsec:traductions_texte_vers_formules}
\begin{itemize}[label=\textbullet]
    \item L'anglais vit dans la maison rouge : $\forall x[\text{anglais}(x)\rightarrow\text{rouge}(x)]$
    \item Le Suédois a des chiens : $\forall x[\text{suédois}(x)\rightarrow\text{chiens}(x)]$
    \item La maison verte est directement à gauche de la maison blanche : \\
    $\forall x\forall y(\text{succgauche}(x,y)\wedge\text{verte}(x))\rightarrow\text{blanche}(y)$
\end{itemize}
\newpage

%================= Bibliography ========================
% \newpage
% \phantomsection % Required if hyperref is used
% \addcontentsline{toc}{section}{References} % Adding bibliography to table of contents
% \printbibliography % Print the bibliography

\end{document}
