\subsection{Introduction}

En logique des prédicats ;
\begin{itemize}[label=\textbullet]
    \item on ajoute les quantificateurs.
    \item on généralise les valeurs que peuvent prendre les variables.
    \item on ajoute des relations poiur décrire certianes relation entre ces valeurs.
    \item on ajoute des symboles de fonctions à la syntaxe.
\end{itemize}
\begin{example}
    $\forall x\forall y \cdot \text{PremierEntreEux}(x,y)\leftrightarrow \exists x'\exists y' \cdot x.x'
    + y.y' = 1$
    \begin{itemize}[label=\textbullet]
        \item PremierEntreEux est un prédicat à deux arguments.
        \item 1 est appelée constante.
        \item $x x' + y y'$ est un terme formé avec les fonctions $\times$ et $.$
    \end{itemize}
\end{example}

\subsection{Syntaxe}
\label{subsec:syntaxe}
\subsubsection{Alphabet}
\label{subsubsec:alphabet}
L'alphabet d'un langage du premier ordre comporte d'abord les symboles suivants qui sont communs à tous ces langages :
\begin{itemize}[label=\textbullet]
    \item les connecteurs : $\neg,\wedge,\vee,\rightarrow\leftrightarrow$;
    \item les paranthèses : $(,)$;
    \item le quantificateur universel $\forall$ et le quantificateur existentiel $\exists$;
    \item un ensemble infini $\vee$ de symboles de variables $x,y,z,\dots$;
\end{itemize}
\begin{definition}{Langage de la logique du premier ordre}{lg_1_ordre}
    Un langgage $\mathcal{L}$ de la logique du premier ordre est caractérisé par :
    \begin{itemize}[label=\textbullet]
        \item des symboles de relations (prédicats), notés $p,q,r,s,\dots$;
        \item des symboles de fonctions, notés $f,g,h,\dots$;
        \item des symboles de constantes, notés $c,d,e,\dots$;
    \end{itemize}
    À chaque prédicat $p$, respectivement fonction $f$, on associe un entier strictement positif appelé 
    l'arité de $p$, respectivement de $f$, càd le nombre d'arguments de $p$, respectivement $f$. On notera parfois $p|_n$ 
    et $f|_n$ pour indiquer que $p$ (respectivement $f$) est un symbole de relation (respectivement de fonction) d'arité $n$.\\
    Le prédicat "=" sera toujours présent.
\end{definition}
\begin{example}
    Exemples de langages :
    \begin{itemize}[label=\textbullet]
        \item $\mathcal{L}_1 = \{r|_1,c\}$ contient un prédicat unaire $r$ et une constante $c$;
        \item $\mathcal{L}_2 = \{r|_2,f|_1,g|_2,h|_2,c,d\}$ contient un prédicat binaire $r$, une fonction unaire $f$,
        deux symboles de fonctions binaires $g$ et $h$ et deux constantes $c$ et $d$.\\
    \end{itemize}
\end{example}

\subsubsection{Construction des termes}
\label{subsubsec:construction_termes}
\begin{definition}{Termes d'un langage}{termes_langage}
    L'ensemble des termes d'un langage $\mathcal{L}$ est le plus petit ensemble qui contient les symboles de constantes et de
    variables et qui est clos par application des fonction.\\
    L'ensemble des termes, noté $\mathcal{T}$, est le plus petit ensemble satisfaisant :
    \begin{enumerate}
        \item Tout symbole de constante ou variable est un terme.
        \item Si $f$ est un symbole de fonction d'arité $n$ et $t_1,t_2,\cdots,t_n$ sont des termes alors $f(t_1,t_2,\cdots,t_n)$ est un terme.
        est un terme.
    \end{enumerate}
\end{definition}
\begin{example}
    Voici des exemples :
    \begin{itemize}[label=\textbullet]
        \item Les seuls termes du langages $\mathcal{L}_1$ sont la constante $c$ et les variables.
        \item Les expressions suivantes sont des termes du langage $\mathcal{L}_2$ : 
        \begin{itemize}[label=\textbullet]
            \item $f(c)$
            \item $f(h(f(c),d))$
            \item $f(y)$
            \item $f(h(f(x),f(d)))$
        \end{itemize}
    \end{itemize}
\end{example}
\noindent Un terme est \textbf{clos} s'il est sans variable. Ici, $f(c)$ est clos.

\subsubsection{Construction des formules}
\label{subsubsec:construction_formules}
\begin{definition}{Formules d'un langage}{formules_langage}
    L'ensemble des formules du langage $\mathcal{L}$, que l'on désigne par $\mathcal{F(L)}$, est défini par la grammaire suivante:
    \begin{equation*}
        \phi ::= p(t_1,\cdots,t_n)|\phi\wedge\phi|\phi\vee\phi|\neg\phi|\phi\rightarrow\phi|\phi\leftrightarrow\phi|\exists x
        \cdot\phi|\forall x\cdot\phi|(\phi)
    \end{equation*}
    \begin{itemize}[label=\textbullet]
        \item $t_1,\cdots,t_n$ sont des termes;
        \item $p$ est un symbole de relation;
        \item $\exists x$ est le quantificateur existentiel;
        \item $\forall x$ est le quantificateur universel;
    \end{itemize}
\end{definition}
\begin{example}
    Voici des exemples :
    \begin{itemize}[label=\textbullet]
        \item La formule $r(c)\vee\neg\exists x\cdot r(x)$ est une formule du langage $\mathcal{L}_1$.
        \item Exemples de formules du langage $\mathcal{L}_2$ :
        \begin{itemize}[label=\textbullet]
            \item $\forall x\cdot\exists y(g(x,y) = c\wedge g(x,y) = c)$
            \item $\forall x\cdot\neg(f(x)=c)$
        \end{itemize}
    \end{itemize}
\end{example}

\subsubsection{Règles de précédence}
\label{subsubsec:règles_précédence}
Pour les Booléens, ce sont les mêmes règles de précédence que dans la logique propositionnelle (cf. \ref{sec:logique_propositionnelle}).
Les quantificateurs ont la même priorité que la négation.
\begin{equation*}
    \forall x\cdot\neg p(x,y)\vee p(y,x) \equiv (\forall x\cdot\neg(p(x,y)))\vee p(y,x)
\end{equation*}

\subsubsection{Variables libres et liées}
\label{subsubsec:variables_libres_liées}
\begin{definition}{Occurence de variable}{occurence_variable}
    Une occurence d'une variable dans une formule est un couple constitué de cette variable et d'une place effective, càd qui
    ne suit pas un quantificateur. 
\end{definition}
\begin{example}
    Dans la formule :
    \begin{equation*}
        r(x,z)\rightarrow\forall z\cdot (r(y,z)\vee y=z)
    \end{equation*}
    La varible $x$ possède une occurence, la variable $y$ deux et la variable $z$ trois.
\end{example}
\begin{definition}{Variables libres ou liées}{variables_libres_liées}
    \begin{itemize}[label=\textbullet]
        \item Une occurence d'une variable $x$ dans une formule $\phi$ est une occurence \textbf{libre} si elle ne se trouve 
        dans aucune sous-formule de $\phi$, qui commence par une quantification $\forall x$ ou $\exists x$.
        \item Dans le cas contraire, l'occurence est dite \textbf{liée}.
        \item Une variable est libre dans une formule si elle a au moins une occurence libre dans cette formule.
        \item Une formule est close est une formule sans variable libre.
        \item On note $\text{Libres}(\phi)$ l'ensemble des variables libres de $\phi$.
    \end{itemize}
\end{definition}
\begin{example}
    Dans $\exists x\cdot p(x,y)$, l'occurence de $x$ est liée et celle de $y$ est libre.
\end{example}



\subsection{Sémantique}
\label{subsec:sémantique}

\subsubsection{Interprétation des formules}
\label{subsubsec:interprétation_formules}

\begin{definition}{Structure}{structure}
    Une structure $\mathcal{M}$ pour un langage $\mathcal{L}$ se compose d'un ensemble non vide $M$ appelé le domaine et d'une 
    interprétation des symboles de prédicats par des relations sur $M$, des symboles de fonctions par des fonctions de $M$ et 
    des constantes par des éléments de $M$.\\
    Plus précisément, une structure est composée de :
    \begin{itemize}[label=\textbullet]
        \item d'un sous-ensemble de $M^n$, noté $p^\mathcal{M}$, pour chaque symbole de prédicat $r$ d'arité $n$ dans $\mathcal{L}$;
        \item d'une fonction \textbf{totale} de $M^m$ dans $M$, notée $f^\mathcal{M}$, pour chaque symbole de fonction $f$ d'arité 
        $m$ dans $\mathcal{L}$;
        \item d'un élément de $M$, noté $c^\mathcal{M}$, pour chaque symbole de constante $c$ dans $\mathcal{L}$.
    \end{itemize}
\end{definition}
\begin{example}
    \begin{itemize}
        \item Pour le langage $\mathcal{L}_1 = (r|_1c)$, la structure $\mathcal{M}_1 = (\mathbb{N}, r^{\mathcal{M}_1}, c^{\mathcal{m}_1})$ 
        avec $r^{\mathcal{M}_1}$ l'ensemble des nombres premiers et $c^{\mathcal{M}_1} = 2$ est une interprétation de $\mathcal{L}_1$.
        \item Pour le langage $\mathcal{L}_2 = (r|_2,f|_1,g|_2,h|_2,c,d)$, on peut prendre la structure sur les réels :
        \begin{equation*}
            \mathcal{M}_2 = (\mathbb{R}, \leq, +1,+,\times,0,1)
        \end{equation*}
        avec la fonction $+1$ qui à $x$ associe $x+1$.
    \end{itemize}
\end{example}
\begin{theorem}{Formule satisfaite}{formule_satisfaite}
    Une formule $\phi$ construite sur un langage $\mathcal{L}$ est satisfaite dans une structure $\mathcal{M}$ et pour une valuation
    $v$ donnanat une valeur aux variables de l'ensemble $\mathcal{V}$ (noté $\mathcal{M}, v\vDash\emptyset$) ssi :
    \begin{itemize}[label=\textbullet]
        \item si $\phi\equiv r(t_1,t_2,\cdots,t_n)$ et $t_i^{\mathcal{M},v}=b_i$ pour $i=1,\cdots,n$, alors $\phi$ est vraie
        ssi $(b_1,b_2,\cdots,b_n)\in r^{\mathcal{M}}$;
        \item si $\phi\equiv\neg\psi_1, \phi\equiv\psi_1\vee\psi_2,\phi\equiv\psi_1\wedge\psi2,\phi\equiv\psi_1\rightarrow\psi_2,
        \phi\equiv\psi_1\leftrightarrow\psi_2$ alors la valeur de $\phi$ est calculée àpd valeurs de $psi_1$ et $psi_2$ comme
        dans le cas propositionnel.
        \item si $\phi\equiv\exists x\cdot\psi$, alors $\phi$ est vraie ssi \textbf{il existe} une valuation $v'$ telle que $\mathcal{M},v'\vDash
        \psi$ et $v'$ est d'accord (=$v'(x)=v(x)$) avec $v$ sur Libres($phi$).
        \item si $\phi\equiv\forall x\cdot\psi$, alors $\phi$ est vraie ssi \textbf{pour toute} valuation $v'$ qui est d'accord avec $v$ sur
        Libres($\phi$), on a $\mathcal{M},v'\vDash\psi$.
    \end{itemize}
\end{theorem}
Lorsque $\mathcal{M},v\vDash\emptyset$, on dit que $\mathcal{M},v$ satisfait $\phi$ ou encore que ($\mathcal{M},v$) est un modèle de $\phi$.
De pus, lorsque $\phi$ est une formule close, alors sa valeur de vérité dans un couple ($\mathcal{M},v$), ne dépend pas de $v$.
On omettera de mentionner $v$ dans ce cas.
\begin{example}
    \begin{itemize}[label=\textbullet]
        \item Prenons $\mathcal{L}_1=\{r|_2,c\}$. La formule close suivante :
        \begin{equation*}
            \begin{aligned}
                &\forall x\cdot r(x,x)\\
                \wedge&\forall x\cdot\forall y\cdot(r(x,y)\rightarrow r(y,x))\\
                \wedge&\forall x\cdot\forall y\cdot\forall z\cdot(r(x,y)\wedge r(y,z)\rightarrow r(x,z))
            \end{aligned}
        \end{equation*}
        exprime qu'une structure $(D,R,a)$ est un modèle de la formule ssi $R$ est une relation d'équivalence.
        \item est-ce que $\exists x\cdot \forall y\cdot r(x,y)$ est vraie dans ($\mathbb{N},\leq$)?\\
        "Est-ce qu'il existe un entier $x\in\mathbb{N}$ tel que $\forall y\in\mathbb{N}$, $x\leq y$?" oui, $x=0$.
        
    \end{itemize}
\end{example}

\subsubsection{Interprétation des termes dans une structure}
\label{subsubsec:interprétation_termes_structure}

\begin{definition}{Valuation}{valuation}
    Etant donnés un ensemble de variables $\mathcal{V}$ et un domaine $M$, une \textit{valuation} pour les variables de $\mathcal{V}$
    dans $M$ est une fonction $v:\mathcal{V}\rightarrow M$ qui attribue à chaque variable $x\in \mathcal{V}$, une valeur $v(x)\in M$.
\end{definition}
\begin{definition}{Interprétation de termes}{inter_terme}
    L'interprétation d'un terme $t$ (dont les variables sont dans $\mathcal{V}$) dans une structure de domaine $M$ et selon une 
    valuation $v$ est un élément $t^{\mathcal{M},v}\in M$, inductivement défini de la façon suivante :
    \begin{itemize}[label=\textbullet]
        \item si $t=c$ alors $t^{\mathcal{M},v} = c^{\mathcal{M}}$;
        \item si $t=x$ alors $t^{\mathcal{M},v} = v(x)$ est $v(x)$;
        \item si $t=f(t_1,\cdots,t_n)$ alors $t^{\mathcal{M},v} = f^{\mathcal{M}}(t_1^{\mathcal{M},v},\cdots,t_n^{\mathcal{M},v})$.
    \end{itemize}
\end{definition}
\begin{example}
    Soit $\mathcal{L}_2 = (r|_2,f|_1,g|_2,h|_2,c,d)$ et $\mathcal{M}_3 = (\mathbb{N},\leq,+1,+,\times,0,1)$.\\
    L'interprétation dans $\mathcal{M}_3$ du terme 
    \begin{equation*}
        t_1 \equiv g(y,h(c,x))
    \end{equation*}
    selon la valuation $v$ telle que $v(x)=3, v(y)=4,v(z)=6$ est :
    \begin{equation*}
        t_1^{\mathcal{M}_3,v} = 4 + (0\times 3) = 4
    \end{equation*}
    L'interprétation du terme 
    \begin{equation*}
        t_2\equiv f(g(d,h(y,z)))
    \end{equation*}
    est :
    \begin{equation*}
        t_2^{\mathcal{M}_3,v} = 1 + (4\times 6) + 1 = 26
    \end{equation*}
\end{example}

\subsubsection{Formules satisfaisables, valides et équivalentes}
\label{subsubsec:formules_satisfaisables_valides_équivalentes}
\begin{enumerate}
    \item Une formule $\phi$ close est \textbf{satisfaisable} si elle un modèle. (il n'existe pas d'algo pour vérifier la satisfaisabilité
    d'une formule).
    \item Une formule $\phi$ close est \textbf{valide} si toutes les structures sont des modèles de $\phi$.
    \item Deux formules $\phi_1,\phi_2$ telles que Libres($\phi_1$) = Libres($\phi_2$) sont \textbf{équivalentes} si la formule
    $\forall x_1\cdots\forall x_n(\phi_1\leftrightarrow\phi_2)$ est valide, avec $\{x_1,\cdots,x_n\}$=Libres($\phi_1$).
    \begin{example}
        Voici quelques exemples de formules équivalentes :
        \begin{itemize}[label=\textbullet]
            \item $\forall x\cdot(\phi\wedge\psi)$ et $(\forall x\cdot\phi)\wedge(\forall x\cdot\psi)$
            \item $\exists x\cdot(\phi\vee\psi)$ et $(\exists x\cdot\phi)\vee(\exists x\cdot\psi)$
        \end{itemize}
    \end{example}
\end{enumerate}

\subsubsection{Exemples de traductions de texte vers formules}
\label{subsubsec:traductions_texte_vers_formules}
\begin{itemize}[label=\textbullet]
    \item L'anglais vit dans la maison rouge : $\forall x[\text{anglais}(x)\rightarrow\text{rouge}(x)]$
    \item Le Suédois a des chiens : $\forall x[\text{suédois}(x)\rightarrow\text{chiens}(x)]$
    \item La maison verte est directement à gauche de la maison blanche : \\
    $\forall x\forall y(\text{succgauche}(x,y)\wedge\text{verte}(x))\rightarrow\text{blanche}(y)$
\end{itemize}